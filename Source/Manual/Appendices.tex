
\chapter{Python} 

\section{Overview}
Python is a widely used general-purpose, high-level programming language. Its design philosophy emphasizes code readability, and its syntax allows programmers to express concepts in fewer lines of code than would be possible in languages such as C. The language provides constructs intended to enable clear programs on both a small and large scale.

\vpara
Python supports multiple programming paradigms, including object-oriented, imperative and functional programming or procedural styles. It features a dynamic type system and automatic memory management and has a large and comprehensive standard library.

\vpara
Like other dynamic languages, Python is often used as a scripting language, but is also used in a wide range of non-scripting contexts. Using third-party tools, Python code can be packaged into standalone executable programs. Python interpreters are available for many operating systems.




\newpage
\section{CPython}

CPython, the reference implementation of Python, is free and open source software and has a community-based development model, as do nearly all of its alternative implementations. CPython is managed by the non-profit Python Software Foundation.

\vpara
For further information on Python, see \href{http://en.wikipedia.org/wiki/Python_(programming_language)}{Wikipedia: Python} (the text above has been copied from this reference), or the  \href{http://www.python.org/}{Python Homepage}. Support for COM is included in the distribution of the \href{http://www.activestate.com/activepython/downloads}{ActivePython Community Edition}.

\vpara
Python can use  GMP und MPFR thanks to \href{http://code.google.com/p/gmpy/}{GMPY2}, with documentation \href{https://gmpy2.readthedocs.org/en/latest/}{here}.

\vpara
IPython is an integrations platform for various scientific libraries (NumPy, SciPy, matlibplot, pandas etc.) \href{http://ipython.org/}{http://ipython.org/}. Popular distributions are the 

Community Edition of Anaconda: \href{http://docs.continuum.io/anaconda/index.html}{http://docs.continuum.io/anaconda/index.html}, 

%EPDFree (aka Canopy Express): \href{https://www.enthought.com/products/epd/free/}{https://www.enthought.com/products/epd/free/} 
%
%and Python(x, y): \href{https://code.google.com/p/pythonxy/}{https://code.google.com/p/pythonxy/}. 

\vpara
Book recommendation: \cite{McKinney2012}.



To compile the mpmath library libraries, Python 2.7 is required.

\subsection{Downloading and installing  CPython 2.7}
ActivePython is ActiveState's complete and ready-to-install distribution of Python. It provides a one-step installation of all essential Python modules, as well as extensive documentation. 
The Windows distribution ships with PyWin32 -- a suite of Windows tools developed by Mark Hammond, including bindings to the Win32 API and Windows COM. 
ActivePython can be downloaded from

\vpara
\href{http://www.activestate.com/activepython/downloads}{http://www.activestate.com/activepython/downloads}.

\vpara
The latest release version of the 2.7x series is 2.7.6.9. You need to download 2 separate files to support compilation of both 32 bit and 64 bit dlls.




%
%
%\newpage
%\subsection{Plotting}
%
%If matplotlib is available, the functions plot and cplot in mpmath can be used to plot functions respectively as x-y graphs and in the complex plane. Also, splot can be used to
%produce 3D surface plots.
%
%\subsubsection{Function curve plots}
%
%\begin{figure}[ht]
%	\centering
%	\includegraphics[scale=1.0]{Charts/png/MpMathPlot.png}
%	\caption{MpMath 2Dplot}
%	\label{Fig MpMath 2Dplot}
%\end{figure}
%
%
%Output of plot([cos, sin], [-4, 4])
%
%\vpara
%mpmath.plot(ctx, f, xlim=[-5, 5], ylim=None, points=200, file=None, dpi=None, singularities=[], axes=None)
%
%\vpara
%Shows a simple 2D plot of a function $f(x)$ or list of functions $[f_0(x),f_1(x),\ldots,f_n(x)]$ over a given interval specified by xlim. Some examples:
%
%\lstset{language={Python}}
%\begin{lstlisting}
%plot(lambda x: exp(x)*li(x), [1, 4])
%plot([cos, sin], [-4, 4])
%plot([fresnels, fresnelc], [-4, 4])
%plot([sqrt, cbrt], [-4, 4])
%plot(lambda t: zeta(0.5+t*j), [-20, 20])
%plot([floor, ceil, abs, sign], [-5, 5])
%\end{lstlisting}
%
%
%Points where the function raises a numerical exception or returns an infinite value are removed from the graph. Singularities can also be excluded explicitly as follows (useful
%for removing erroneous vertical lines):
%
%\lstset{language={Python}}
%\begin{lstlisting}
%plot(cot, ylim=[-5, 5]) # bad
%plot(cot, ylim=[-5, 5], singularities=[-pi, 0, pi]) # good
%\end{lstlisting}
%
%
%For parts where the function assumes complex values, the real part is plotted with dashes and the imaginary part is plotted with dots.
%
%Note: This function requires matplotlib (pylab).
%
%\newpage
%\subsubsection{Complex function plots}
%
%\begin{figure}[ht]
%	\centering
%	\includegraphics[scale=1.0]{Charts/png/MpMathCplot.png}
%	\caption{MpMath cplot}
%	\label{Fig MpMath cplot}
%\end{figure}
%
%
%
%Output of fp.cplot(fp.gamma, points=100000)
%
%\vpara
%mpmath.cplot(ctx, f, re=[-5, 5], im=[-5, 5], points=2000, color=None, verbose=False, file=None, dpi=None, axes=None)
%
%\vpara
%Plots the given complex-valued function f over a rectangular part of the complex plane specified by the pairs of intervals re and im. For example:
%
%\lstset{language={Python}}
%\begin{lstlisting}
%cplot(lambda z: z, [-2, 2], [-10, 10])
%cplot(exp)
%cplot(zeta, [0, 1], [0, 50])
%\end{lstlisting}
%
%
%By default, the complex argument (phase) is shown as color (hue) and the magnitude is show as brightness. You can also supply a custom color function (color). This function should take a complex number as input and return an RGB 3-tuple containing floats in the range 0.0-1.0.
%
%\vpara
%To obtain a sharp image, the number of points may need to be increased to 100,000 or thereabout. Since evaluating the function that many times is likely to be slow, the 'verbose' option is useful to display progress.
%
%Note: This function requires matplotlib (pylab).
%
%
%\newpage
%\subsubsection{3D surface plots}
%
%
%Output of splot for the donut example.
%
%\vpara
%mpmath.splot(ctx, f, u=[-5, 5], v=[-5, 5], points=100, keep\_aspect=True, wireframe=False, file=None, dpi=None, axes=None)
%
%\begin{figure}[ht]
%	\centering
%	\includegraphics[scale=1.0]{Charts/png/MpMathSplot.png}
%	\caption{MpMath Surface plot}
%	\label{Fig MpMath Surface plot}
%\end{figure}
%
%
%Plots the surface defined by $f$.
%
%\vpara
%If $f$ returns a single component, then this plots the surface defined by $z=f(x,y)$ over the rectangular domain with $x=u$  and $y=v$.
%
%\vpara
%If $f$ returns three components, then this plots the parametric surface $x,y,z = f(u,v)$ over the pairs of intervals $u$ and $v$.
%
%\vpara
%For example, to plot a simple function:
%
%\lstset{language={Python}}
%\begin{lstlisting}
%>>> from mpmath import *
%>>> f = lambda x, y: sin(x+y)*cos(y)
%>>> splot(f, [-pi,pi], [-pi,pi])
%\end{lstlisting}
%
%
%Plotting a donut:
%
%\lstset{language={Python}}
%\begin{lstlisting}
%>>> r, R = 1, 2.5
%>>> f = lambda u, v: [r*cos(u), (R+r*sin(u))*cos(v), (R+r*sin(u))*sin(v)]
%>>> splot(f, [0, 2*pi], [0, 2*pi])
%\end{lstlisting}
%
%
%Note: This function requires matplotlib (pylab) 0.98.5.3 or higher.


\subsection{Using the C-API}

This section describes how to write modules in C or C++ to extend the Python interpreter with new modules. Those modules can not only define new functions but also new object types and their methods. The document also describes how to embed the Python interpreter in another application, for use as an extension language. Finally, it shows how to compile and link extension modules so that they can be loaded dynamically (at run time) into the interpreter, if the underlying operating system supports this feature.

This document assumes basic knowledge about Python. For an informal introduction to the language, see The Python Tutorial. The Python Language Reference gives a more formal definition of the language. The Python Standard Library documents the existing object types, functions and modules (both built-in and written in Python) that give the language its wide application range.

For a detailed description of the whole Python/C API, see the separate Python/C API Reference Manual.

. Extending Python with C or C++

It is quite easy to add new built-in modules to Python, if you know how to program in C. Such extension modules can do two things that can’t be done directly in Python: they can implement new built-in object types, and they can call C library functions and system calls.

To support extensions, the Python API (Application Programmers Interface) defines a set of functions, macros and variables that provide access to most aspects of the Python run-time system. The Python API is incorporated in a C source file by including the header "Python.h".

The compilation of an extension module depends on its intended use as well as on your system setup; details are given in later chapters.

Note:
The C extension interface is specific to CPython, and extension modules do not work on other Python implementations. In many cases, it is possible to avoid writing C extensions and preserve portability to other implementations. For example, if your use case is calling C library functions or system calls, you should consider using the ctypes module or the cffi library rather than writing custom C code. These modules let you write Python code to interface with C code and are more portable between implementations of Python than writing and compiling a C extension module.

As an example for an extension module which provides additional functionality in multi-precision computing, see the documentation on gmpy2 (section \ref{gmpy2}).





\subsection{Interfaces to the C family of languages}

\subsubsection{Windows, GNU/Linux, Mac OSX: GNU Compiler Collection}
The GNU Compiler Collection (GCC) is a compiler system produced by the GNU Project supporting various programming languages. GCC is a key component of the GNU toolchain. The Free Software Foundation (FSF) distributes GCC under the GNU General Public License (GNU GPL). GCC has played an important role in the growth of free software, as both a tool and an example.

\vpara
Originally named the GNU C Compiler, because it only handled the C programming language, GCC 1.0 was released in 1987 and the compiler was extended to compile C++ in December of that year.[1] Front ends were later developed for Objective-C, Objective-C++, Fortran, Java, Ada, and Go among others.[3]

\vpara
As well as being the official compiler of the unfinished GNU operating system, GCC has been adopted as the standard compiler by most other modern Unix-like computer operating systems, including Linux and the BSD family. A port to RISC OS has also been developed extensively in recent years. There is also an old (3.0) port of GCC to Plan9, running under its ANSI/POSIX Environment (APE).[4] GCC is also available for Microsoft Windows operating systems and for the ARM processor used by many portable devices.

\vpara
For further information on the GNU Compiler Collection, see \href{http://en.wikipedia.org/wiki/GNU_Compiler_Collection}{Wikipedia: GCC} (the text above has been copied from this reference), or the  \href{http://gcc.gnu.org/}{GCC Homepage}.


\subsubsection{Windows: MSVC}
Microsoft Visual C++ (often abbreviated as MSVC or VC++) is a commercial (free version available), integrated development environment (IDE) product from Microsoft for the C, C++, and C++/CLI programming languages. It features tools for developing and debugging C++ code, especially code written for the Microsoft Windows API, the DirectX API, and the Microsoft .NET Framework.

\vpara
Although the product originated as an IDE for the C programming language, the compiler's support for that language conforms only to the original edition of the C standard, dating from 1989. The later revisions of the standard, C99 and C11, are not supported.[41] According to Herb Sutter, the C compiler is only included for "historical reasons" and is not planned to be further developed. Users are advised to either use only the subset of the C language that is also valid C++, and then use the C++ compiler to compile their code, or to just use a different compiler such as Intel C++ Compiler or the GNU Compiler Collection instead.[42]

\vpara
For further information on Microsoft Visual C++, see \href{http://en.wikipedia.org/wiki/Visual_C\%2B\%2B}{Wikipedia: MSVC} (the text above has been copied from this reference), or the  \href{http://msdn.microsoft.com/en-us/vstudio/hh386302}{MSVC Homepage}.


\vpara
The following C file makes a number of direct calls into Python, passing arguments as strings and receiving a result as string:

\lstset{language={C++}}
\begin{lstlisting}

#include "stdafx.h"
#include "CallPython.h"

int main(int argc, const char *argv[])
{
long ResultLong = SetSpecialValue_Long(2,3,4);
printf( "This is a long: %d\n",  ResultLong);

double ResultDouble = SetSpecialValue_Double(2.0,3.0,4.0);
printf( "This is a double: %f\n",  ResultDouble);

const char *sLong2[] = {"TestLong", "lll", "3", "1432"};
MyPythonFunction(4, sLong2);

const char *sDouble2[] = {"TestDouble", "fff", "13.5", "265.34"};
MyPythonFunction(4, sDouble2);

const char *sString2[] = {"TestStringFunc", "sss", "3", "2"};
MyPythonFunction(4, sString2);

const char *sString3[] = {"TestStringMpf2", "3", "2.456"};
MyPythonFunctionString2(3, sString3);

char buffer[1600]; // 1600 bytes allocated here on the stack.
int size0a = MyPythonFunctionStringReturn(3, sString3, buffer, sizeof(buffer));
printf("New New0a %s\n", buffer); // prints "Mar"
//printf("Length of string: %ld\n", size0a);

char buffer0[1600]; // 1600 bytes allocated here on the stack.
int size0 = MyPythonFunctionStringReturn00("TestStringMpf0", buffer0, sizeof(buffer0));
printf("New New0 %s\n", buffer0); // prints "Mar"
//printf("Length of string0: %ld\n", size0);

char buffer1[1600]; // 1600 bytes allocated here on the stack.
int size1 = MyPythonFunctionStringReturn01("TestStringMpf1", "3", buffer1, sizeof(buffer1));
printf("New New1 %s\n", buffer1); // prints "Mar"
//printf("Length of string: %ld\n", size1);

char buffer2[1600]; // 1600 bytes allocated here on the stack.
int size2 = MyPythonFunctionStringReturn02("TestStringMpf2", "3", "2.456", buffer2, sizeof(buffer2));
printf("New New2 %s\n", buffer2); // prints "Mar"
//printf("Length of string: %ld\n", size2);

ClosePy();
return 0;
}

\end{lstlisting}

The header file CallPython.h looks like this:

\begin{lstlisting}
#pragma warning(disable: 4244)

#ifdef CALLPYTHON_EXPORTS
#define MPNUMC_DLL_IMPORTEXPORT __declspec(dllexport)
#else
#define MPNUMC_DLL_IMPORTEXPORT __declspec(dllimport)
#endif

#ifdef __cplusplus
extern "C" {
#endif

MPNUMC_DLL_IMPORTEXPORT long SetSpecialValue_Long(long m, long n, long what);
MPNUMC_DLL_IMPORTEXPORT double SetSpecialValue_Double(double m, double n, double what);
MPNUMC_DLL_IMPORTEXPORT int CallPythonFunction(int argc, const char *argv[]);
MPNUMC_DLL_IMPORTEXPORT int MyPythonFunction(int argc, const char *argv[]);
MPNUMC_DLL_IMPORTEXPORT int MyPythonFunctionString(int argc, const char *argv[]);
MPNUMC_DLL_IMPORTEXPORT int MyPythonFunctionString2(int argc, const char *argv[]);
MPNUMC_DLL_IMPORTEXPORT int MyPythonFunctionStringReturn(int argc, const char *argv[], char* buffer, int buffersize);

MPNUMC_DLL_IMPORTEXPORT int MyPythonFunctionStringReturn00(const char* FuncName, char* buffer, int buffersize);
MPNUMC_DLL_IMPORTEXPORT int MyPythonFunctionStringReturn01(const char* FuncName, const char* Arg01, char* buffer, int buffersize);
MPNUMC_DLL_IMPORTEXPORT int MyPythonFunctionStringReturn02(const char* FuncName, const char* Arg01, const char* Arg02, char* buffer, int buffersize);

MPNUMC_DLL_IMPORTEXPORT void ClosePy();
#ifdef __cplusplus
}
#endif
\end{lstlisting}


The C file CallPython.cpp (which produces the dynamic link library) looks like this:

\lstset{language={C++}}
\begin{lstlisting}

#define _CRT_SECURE_NO_WARNINGS
#include "CallPython.h"
#include "stdafx.h"
#include<stdio.h>
#include <Python.h>

long SetSpecialValue_Long(long m, long n, long what)
{
return (m + n + 1) * what;
}

double SetSpecialValue_Double(double m, double n, double what)
{
return (m + n) * what;
}


int MyPythonFunctionStringReturn00(const char* FuncName, char* buffer, int buffersize)
{
PyObject *pFunc;
PyObject *pValue;
Py_ssize_t size;
PyObject *pModule= GetPythonModule2();
pFunc = PyObject_GetAttrString(pModule, FuncName);
if (pFunc && PyCallable_Check(pFunc)) {
pValue = PyObject_CallObject(pFunc, NULL);
if (pValue != NULL) {
strncpy(buffer, PyUnicode_AsUTF8AndSize(pValue, &size), buffersize-1);
Py_DECREF(pValue);
}
}
Py_XDECREF(pFunc);
return size;
}


int MyPythonFunctionStringReturn01(const char* FuncName, const char* Arg01, char* buffer, int buffersize)
{
PyObject *pFunc, *pArgs, *pValue;
PyObject *pModule= GetPythonModule2();
Py_ssize_t size;
pFunc = PyObject_GetAttrString(pModule, FuncName);
if (pFunc && PyCallable_Check(pFunc)) {
pArgs = PyTuple_New(1);
pValue = PyUnicode_FromString(Arg01);
PyTuple_SetItem(pArgs, 0, pValue);

pValue = PyObject_CallObject(pFunc, pArgs);
Py_DECREF(pArgs);
if (pValue != NULL) {
strncpy(buffer, PyUnicode_AsUTF8AndSize(pValue, &size), buffersize-1);
Py_DECREF(pValue);
}
}
Py_XDECREF(pFunc);
return size;
}


int MyPythonFunctionStringReturn02(const char* FuncName, const char* Arg01, const char* Arg02, char* buffer, int buffersize)
{
PyObject *pFunc, *pArgs, *pValue;
PyObject *pModule= GetPythonModule2();
Py_ssize_t size;

pFunc = PyObject_GetAttrString(pModule, FuncName);

if (pFunc && PyCallable_Check(pFunc)) {
pArgs = PyTuple_New(2);
pValue = PyUnicode_FromString(Arg01);
PyTuple_SetItem(pArgs, 0, pValue);

pValue = PyUnicode_FromString(Arg02);
PyTuple_SetItem(pArgs, 1, pValue);

pValue = PyObject_CallObject(pFunc, pArgs);
Py_DECREF(pArgs);
if (pValue != NULL) {
strncpy(buffer, PyUnicode_AsUTF8AndSize(pValue, &size), buffersize-1);
Py_DECREF(pValue);
}
}
Py_XDECREF(pFunc);
return size;
}

void ClosePy()
{
PyObject *pModule;
pModule = GetPythonModule2();
Py_DECREF(pModule);
Py_Finalize();
}
\end{lstlisting}




\newpage
\subsection{Cython: C extensions for the Python language}

[Cython] is a programming language that makes writing C extensions for the Python language as easy as Python itself. It aims to become a superset of the [Python] language which gives it high-level, object-oriented, functional, and dynamic programming. Its main feature on top of these is support for optional static type declarations as part of the language. The source code gets translated into optimized C/C++ code and compiled as Python extension modules. This allows for both very fast program execution and tight integration with external C libraries, while keeping up the high programmer productivity for which the Python language is well known.

The primary Python execution environment is commonly referred to as CPython, as it is written in C. Other major implementations use Java (Jython [Jython]), C\# (IronPython [IronPython]) and Python itself (PyPy [PyPy]). Written in C, CPython has been conducive to wrapping many external libraries that interface through the C language. It has, however, remained non trivial to write the necessary glue code in C, especially for programmers who are more fluent in a high-level language like Python than in a close-to-the-metal language like C.

Originally based on the well-known Pyrex [Pyrex], the Cython project has approached this problem by means of a source code compiler that translates Python code to equivalent C code. This code is executed within the CPython runtime environment, but at the speed of compiled C and with the ability to call directly into C libraries. At the same time, it keeps the original interface of the Python source code, which makes it directly usable from Python code. These two-fold characteristics enable Cython’s two major use cases: extending the CPython interpreter with fast binary modules, and interfacing Python code with external C libraries.

While Cython can compile (most) regular Python code, the generated C code usually gains major (and sometime impressive) speed improvements from optional static type declarations for both Python and C types. These allow Cython to assign C semantics to parts of the code, and to translate them into very efficient C code. Type declarations can therefore be used for two purposes: for moving code sections from dynamic Python semantics into static-and-fast C semantics, but also for directly manipulating types defined in external libraries. Cython thus merges the two worlds into a very broadly applicable programming language.



\subsection{A Windows-specific interface: using COM}
\vpara
Example for using the library

\lstset{language={Python}}
\begin{lstlisting}
#Enable COM support
from win32com.client import Dispatch

#Load the mpNumerics library
mp = Dispatch("mpNumerics.mp_Lib")

#Set Floating point type to MPFR with 60 decimal digits precision
mp.FloatingPointType = 3
mp.Prec10 = 60

#Assign values to x1 and x2
x1 = mp.Real(4.5)
x2 = mp.Real(1.21)

#Calculate x3 = x1 / x2
x3 = x1.Div(x2)

#Print the value of x3
print (x3.Str())
\end{lstlisting}

\vpara
Example for using Excel

\lstset{language={Python}}
\begin{lstlisting}
#Enable COM support
from win32com.client import Dispatch

#Load the Excel library
xl = Dispatch("Excel.Application")
xl.Visible = 1
xl.Workbooks.Add() 
xl.Cells(1,1).Value = "Hello442" 
print("From Python")
\end{lstlisting}



%
%\newpage
%\section{PyPy: a fast alternative Python interpreter}
%
%
%PyPy is a fast, compliant alternative implementation of the Python language (2.7.8 and 3.2.5). Thanks to its Just-in-Time compiler, Python programs often run faster on PyPy than on CPython.
%
%PyPy’s Python Interpreter is written in RPython and implements the full Python language. This interpreter very closely emulates the behavior of CPython. It contains the following key components:
%•a bytecode compiler responsible for producing Python code objects from the source code of a user application;
%•a bytecode evaluator responsible for interpreting Python code objects;
%•a standard object space, responsible for creating and manipulating the Python objects seen by the application.
%
%The bytecode compiler is the preprocessing phase that produces a compact bytecode format via a chain of flexible passes (tokenizer, lexer, parser, abstract syntax tree builder, bytecode generator). The bytecode evaluator interprets this bytecode. It does most of its work by delegating all actual manipulations of user objects to the object space. The latter can be thought of as the library of built-in types. It defines the implementation of the user objects, like integers and lists, as well as the operations between them, like addition or truth-value-testing.
%
%This division between bytecode evaluator and object space gives a lot of flexibility. One can plug in different object spaces to get different or enriched behaviours of the Python objects.
%
%
%\subsection{Installation}
%
%\subsubsection{Windows}
%
%\subsubsection{Mac OSX}
%
%\subsubsection{GNU/Linux}
%
%
%
%\newpage
%\section{Jython: Python for the Java Virtual Machine}
%
%This list of JVM Languages comprises notable computer
%programming languages that are used to produce
%software that runs on the Java Virtual Machine (JVM).
%Some of these languages are interpreted by a Java program,
%and some are compiled to Java bytecode and JITcompiled
%during execution as regular Java programs to
%improve performance.
%The JVM was initially designed to support only the Java
%programming language. However, as time passed, ever
%more languages were adapted or designed to run on the
%Java platform.
%
%Apart from the Java language itself, the most common or
%well-known JVM languages are:
%
%\begin{itemize}		
%	\item Clojure, a Lisp dialect
%	\item Groovy, a programming and scripting language
%	\item Scala, an object-oriented and functional programming language[1]
%	\item JRuby, an implementation of Ruby
%	\item Jython, an implementation of Python
%\end{itemize}
%
%Jython is an implementation of the Python language for the Java platform. Throughout this book, you will be learning how to use the Python language, and along the way we will show you where the Jython implementation differs from CPython, which is the canonical implementation of Python written in the C language. It is important to note that the Python language syntax remains consistent throughout the different implementations. At the time of this writing, there are three mainstream implementations of Python. These implementations are: CPython, Jython for the Java platform, and IronPython for the .NET platform. At the time of this writing, CPython is the most prevalent of the implementations. Therefore if you see the word Python somewhere, it could well be referring to that implementation.
%
%This book will reference the Python language in sections regarding the language syntax or functionality that is inherent to the language itself. However, the book will reference the name Jython when discussing functionality and techniques that are specific to the Java platform implementation. No doubt about it, this book will go in-depth to cover the key features of Jython and you’ll learn concepts that only adhere the Jython implementation. Along the way, you will learn how to program in Python and advanced techniques.
%
%Developers from all languages and backgrounds will benefit from this book. Whether you are interested in learning Python for the first time or discovering Jython techniques and advanced concepts, this book is a good fit. Java developers and those who are new to the Python language will find specific interest in reading through Part I of this book as it will teach the Python language from the basics to more advanced concepts. Seasoned Python developers will probably find more interest in Part II and Part III as they focus more on the Jython implementation specifics. Often in this reference, you will see Java code compared with Python code.
%
%
%\subsection{Installing Java}
%
%\subsubsection{Windows}
%
%\subsubsection{Mac OSX}
%
%\subsubsection{GNU/Linux}
%
%
%
%\subsection{Installing Eclipse with support for JPython}
%
%
%\subsubsection{Windows}
%
%\subsubsection{Mac OSX}
%
%\subsubsection{GNU/Linux}
%
%
%
%
%
%\newpage
%\section{IronPython: Python for .NET Framework and Mono}
%
%
%IronPython is an implementation of the Python programming language targeting the .NET Framework and Mono. Jim Hugunin created the project and actively contributed to it up until Version 1.0 which was released on September 5, 2006.[2] Thereafter, it was maintained by a small team at Microsoft until the 2.7 Beta 1 release; Microsoft abandoned IronPython (and its sister project IronRuby) in late 2010, after which Hugunin left to work at Google.[3] IronPython 2.0 was released on December 10, 2008.[4] The project is currently maintained by a group of volunteers at Microsoft's CodePlex open-source repository. It is free and open-source software, and can be implemented with Python Tools for Visual Studio, which is a free and open-source extension for free, isolated, and commercial versions of Microsoft's Visual Studio IDE.[5] [6]
%
%IronPython is written entirely in C\#, although some of its code is automatically generated by a code generator written in Python.
%
%IronPython is implemented on top of the Dynamic Language Runtime (DLR), a library running on top of the Common Language Infrastructure that provides dynamic typing and dynamic method dispatch, among other things, for dynamic languages.[7] The DLR is part of the .NET Framework 4.0 and is also a part of trunk builds of Mono. The DLR can also be used as a library on older CLI implementations.
%
%For further information, see \href{http://en.wikipedia.org/wiki/IronPython}{Wikipedia: IronPython} (the text above has been copied from this reference).
%
%The original distribution of SharpDevelop includes IronPython (it is not included in the mpFormulaPy IDE). Ironpython can be downloaded from the 
%\href{http://ironpython.net/}{IronPython Homepage}. Visual Studio integration is available through  \href{http://ironpython.net/tools/}{Python Tools for Visual Studio}.
%
%
%\lstset{language={Python}}
%\begin{lstlisting}
%#Load CLR
%import clr
%
%#Load the mpNumerics library
%clr.AddReference('MatrixClass2')
%from MatrixClass2 import mp, mpNum
%
%#Set Floating point type to MPFR with 60 decimal digits precision
%mp.FloatingPointType = 3;
%mp.Prec10 = 60;
%
%#Assign values to x1 and x2
%x1 = mp.CNum("32.47");
%x2 = mp.CNum("12.41");
%
%#Calculate x3 = x1 / x2
%x3 = x1 / x2;
%
%#Print the value of x3
%print "Result: ", x3.Str();
%\end{lstlisting}
%
%
%
%
%\newpage
%\section{Pythonista: Python for iOS}
%
%Pythonista is an integrated development environment for writing Python scripts on iOS. You can create interactive experiments and prototypes using multi-touch, animations, and sound – or just use the interactive prompt as a powerful calculator.
%
%With extensive support for URL schemes, access to the iOS clipboard and photo library, and all the powerful libraries that come with Python, it is also possible to use Pythonista as a flexible automation tool for text or image processing.
%
%Pythonista is also an excellent companion for learning the Python language – You can get started easily with lots of ready-to-run examples, the interactive prompt helps you experiment, and you can read the entire documentation right within the app.
%
%The code editor has everything you'd expect: Syntax highlighting, code completion and an extended keyboard specifically designed for writing Python. You can even extend the editor with your own scripts.
%
%The Standard Library has tons of modules for doing math, processing text, working with data from the web, and much more. Additionally, you have access to the iOS clipboard, an in-app web browser and the powerful Python Imaging Library — including support for accessing photos in your camera roll.
%
%Pythonista includes the NumPy and matplotlib packages for scientific computation and data visualization. 
%
%With the integrated UI Editor, you can create user interfaces for your scripts without writing any code.
%
%
%
%
%
%\newpage
%\section{QPython: Python for Android}
%
%QPython is a script engine that runs Python on android devices. It lets your android device run Python scripts and projects. It contains the Python interpreter, console, editor, and the SL4A Library for Android. It’s Python on Android! It offers the development kit which lets you easily develop Python projects and scripts on your Android device.
%
%
%[[ Main Features ]] 
%
%* Supports Python programming on Android including web apps, games, and SL4A programming etc * Run Python scripts / projects on Android devices 
%
%* Can execute Python code \& files from QRCode 
%
%* QEdit lets you create/edit Python scripts / projects easily 
%
%* Includes many useful python libraries 
%
%* Support pip 
%
%[[ Programming Features ]] 
%
%* Supports Web App programming, which let you develop mobile apps with web development framework, this speeds up your mobile development greatly 
%
%* Supports native UI programming, which let you develop games more easily by using scripts 
%
%* Supports SL4A programming to access Android’s features: network, Bluetooth, GPS, and more
%
%
%
%
%\newpage
%\section{Brython: a Python to JavaScript compiler}
%
%Brython is a Python to JavaScript compiler, but it does the whole job in the browser and so makes Python appear to be a client scripting language. 
%
%Brythonic is an ancient Celtic language, but in this case Brython stands for Browser Python, or I suppose it could be reference to the Life of Brian. Wherever the name derives, Brython is another of the growing examples of treating JavaScript as an assembly language. In this case, though, the approach is slightly different. The idea is that you can write Python in the browser as if it was a language with built-in browser support. 
%
%For example:
%
%\lstset{language={Python}}
%\begin{lstlisting}
%<script type="text/python">
%def echo():
%alert("Hello %s !" %doc["zone"].value) </script>
%\end{lstlisting}
%
%You can see that this looks like a native script. You can also see that it is Python with a few extensions to enable it to work in the browser environment - doc represents the DOM and you can use it to select an element using indexing doc["zone"] is the element with id "zone". 
%
%
%
%




\chapter{Building the library and toolbox} 

Building the toolbox and the library from scratch is a much more involved process than just using them. 

Conceptually, it could be described as a top-down process which starts at the level of the modification of the source files for the documentation, the following (automated) generation of various *.xml, *.cs, *.h files and their processing with appropriate tools, which create the .NET, COM, native DLL and spreadsheet interfaces, ultimately leading to the connecting point of the mpNumC.h header file.

It could also be described as a bottom-up process, starting with the compilation of the *.c, *.h amd *.asm of the GMP, MPFR and FLINT library. followed by the compilation of the Eigen and Boost template libraries with the various supported data types, again leading  to the connecting point of the mpNumC.h header file.

In practice, it is easiest to start any rebuilding of the toolbox or the library with an already working installation, with the following steps in mind:

\begin{itemize}
	\item When changing a function, or introducing a new one, always start at the documentation, and provide all information which is required for automated generation of dependent files.
	\item Compile the documentation in latex, and process the output with makemenu etc.
	\item Run the routines which are necessary to update the .NET, COM, native DLL and spreadsheet interfaces.
	\item Decide whether you need to update the mpNumC.h header file.
\end{itemize}

Alternatively, you could start with a breaking change in one of the underlying libraries (e.g. GMP), recompile them first, then recompile all of the dependent libraries.






\chapter{Acknowledgements} 

\section{Contributors to libraries used in the numerical routines}


\subsection{Contributors to mpMath}
\label{Contributors to mpMath}
The following text has been copied from the mpMath manual (0.19):

\vpara
xxxx



\subsection{Contributors to gmpy2}
\label{Contributors to gmpy2}
The following text has been copied from the gmpy2 manual (2.05):

\vpara
xxxx



\small
\chapter{Licenses} % appeared as appendix B
\section{GNU Licenses}



\subsection{GNU General Public License, Version 2}
\label{GPLv2}
\begin{center}
	GNU LIBRARY GENERAL PUBLIC LICENSE
	
	Version 2, June 1991 
\end{center}


\noindent Copyright (C) 1991 Free Software Foundation, Inc.

51 Franklin St, Fifth Floor, Boston, MA  02110-1301, USA


Everyone is permitted to copy and distribute verbatim copies
of this license document, but changing it is not allowed.

[This is the first released version of the library GPL.  It is
numbered 2 because it goes with version 2 of the ordinary GPL.]
Preamble
The licenses for most software are designed to take away your freedom to share and change it. By contrast, the GNU General Public Licenses are intended to guarantee your freedom to share and change free software--to make sure the software is free for all its users. 

This license, the Library General Public License, applies to some specially designated Free Software Foundation software, and to any other libraries whose authors decide to use it. You can use it for your libraries, too. 

When we speak of free software, we are referring to freedom, not price. Our General Public Licenses are designed to make sure that you have the freedom to distribute copies of free software (and charge for this service if you wish), that you receive source code or can get it if you want it, that you can change the software or use pieces of it in new free programs; and that you know you can do these things. 

To protect your rights, we need to make restrictions that forbid anyone to deny you these rights or to ask you to surrender the rights. These restrictions translate to certain responsibilities for you if you distribute copies of the library, or if you modify it. 

For example, if you distribute copies of the library, whether gratis or for a fee, you must give the recipients all the rights that we gave you. You must make sure that they, too, receive or can get the source code. If you link a program with the library, you must provide complete object files to the recipients so that they can relink them with the library, after making changes to the library and recompiling it. And you must show them these terms so they know their rights. 

Our method of protecting your rights has two steps: (1) copyright the library, and (2) offer you this license which gives you legal permission to copy, distribute and/or modify the library. 

Also, for each distributor's protection, we want to make certain that everyone understands that there is no warranty for this free library. If the library is modified by someone else and passed on, we want its recipients to know that what they have is not the original version, so that any problems introduced by others will not reflect on the original authors' reputations. 

Finally, any free program is threatened constantly by software patents. We wish to avoid the danger that companies distributing free software will individually obtain patent licenses, thus in effect transforming the program into proprietary software. To prevent this, we have made it clear that any patent must be licensed for everyone's free use or not licensed at all. 

Most GNU software, including some libraries, is covered by the ordinary GNU General Public License, which was designed for utility programs. This license, the GNU Library General Public License, applies to certain designated libraries. This license is quite different from the ordinary one; be sure to read it in full, and don't assume that anything in it is the same as in the ordinary license. 

The reason we have a separate public license for some libraries is that they blur the distinction we usually make between modifying or adding to a program and simply using it. Linking a program with a library, without changing the library, is in some sense simply using the library, and is analogous to running a utility program or application program. However, in a textual and legal sense, the linked executable is a combined work, a derivative of the original library, and the ordinary General Public License treats it as such. 

Because of this blurred distinction, using the ordinary General Public License for libraries did not effectively promote software sharing, because most developers did not use the libraries. We concluded that weaker conditions might promote sharing better. 

However, unrestricted linking of non-free programs would deprive the users of those programs of all benefit from the free status of the libraries themselves. This Library General Public License is intended to permit developers of non-free programs to use free libraries, while preserving your freedom as a user of such programs to change the free libraries that are incorporated in them. (We have not seen how to achieve this as regards changes in header files, but we have achieved it as regards changes in the actual functions of the Library.) The hope is that this will lead to faster development of free libraries. 

The precise terms and conditions for copying, distribution and modification follow. Pay close attention to the difference between a "work based on the library" and a "work that uses the library". The former contains code derived from the library, while the latter only works together with the library. 

Note that it is possible for a library to be covered by the ordinary General Public License rather than by this special one. 

\vparasmall
TERMS AND CONDITIONS FOR COPYING, DISTRIBUTION AND MODIFICATION

\vparasmall
0. This License Agreement applies to any software library which contains a notice placed by the copyright holder or other authorized party saying it may be distributed under the terms of this Library General Public License (also called "this License"). Each licensee is addressed as "you". 

A "library" means a collection of software functions and/or data prepared so as to be conveniently linked with application programs (which use some of those functions and data) to form executables. 

The "Library", below, refers to any such software library or work which has been distributed under these terms. A "work based on the Library" means either the Library or any derivative work under copyright law: that is to say, a work containing the Library or a portion of it, either verbatim or with modifications and/or translated straightforwardly into another language. (Hereinafter, translation is included without limitation in the term "modification".) 

"Source code" for a work means the preferred form of the work for making modifications to it. For a library, complete source code means all the source code for all modules it contains, plus any associated interface definition files, plus the scripts used to control compilation and installation of the library. 

Activities other than copying, distribution and modification are not covered by this License; they are outside its scope. The act of running a program using the Library is not restricted, and output from such a program is covered only if its contents constitute a work based on the Library (independent of the use of the Library in a tool for writing it). Whether that is true depends on what the Library does and what the program that uses the Library does. 

\vparasmall
1. You may copy and distribute verbatim copies of the Library's complete source code as you receive it, in any medium, provided that you conspicuously and appropriately publish on each copy an appropriate copyright notice and disclaimer of warranty; keep intact all the notices that refer to this License and to the absence of any warranty; and distribute a copy of this License along with the Library. 

You may charge a fee for the physical act of transferring a copy, and you may at your option offer warranty protection in exchange for a fee. 

\vparasmall
2. You may modify your copy or copies of the Library or any portion of it, thus forming a work based on the Library, and copy and distribute such modifications or work under the terms of Section 1 above, provided that you also meet all of these conditions: 

•a) The modified work must itself be a software library. 
•b) You must cause the files modified to carry prominent notices stating that you changed the files and the date of any change. 
•c) You must cause the whole of the work to be licensed at no charge to all third parties under the terms of this License. 
•d) If a facility in the modified Library refers to a function or a table of data to be supplied by an application program that uses the facility, other than as an argument passed when the facility is invoked, then you must make a good faith effort to ensure that, in the event an application does not supply such function or table, the facility still operates, and performs whatever part of its purpose remains meaningful. 
(For example, a function in a library to compute square roots has a purpose that is entirely well-defined independent of the application. Therefore, Subsection 2d requires that any application-supplied function or table used by this function must be optional: if the application does not supply it, the square root function must still compute square roots.)

These requirements apply to the modified work as a whole. If identifiable sections of that work are not derived from the Library, and can be reasonably considered independent and separate works in themselves, then this License, and its terms, do not apply to those sections when you distribute them as separate works. But when you distribute the same sections as part of a whole which is a work based on the Library, the distribution of the whole must be on the terms of this License, whose permissions for other licensees extend to the entire whole, and thus to each and every part regardless of who wrote it. 

Thus, it is not the intent of this section to claim rights or contest your rights to work written entirely by you; rather, the intent is to exercise the right to control the distribution of derivative or collective works based on the Library. 

In addition, mere aggregation of another work not based on the Library with the Library (or with a work based on the Library) on a volume of a storage or distribution medium does not bring the other work under the scope of this License. 

\vparasmall
3. You may opt to apply the terms of the ordinary GNU General Public License instead of this License to a given copy of the Library. To do this, you must alter all the notices that refer to this License, so that they refer to the ordinary GNU General Public License, version 2, instead of to this License. (If a newer version than version 2 of the ordinary GNU General Public License has appeared, then you can specify that version instead if you wish.) Do not make any other change in these notices. 

Once this change is made in a given copy, it is irreversible for that copy, so the ordinary GNU General Public License applies to all subsequent copies and derivative works made from that copy. 

This option is useful when you wish to copy part of the code of the Library into a program that is not a library. 

\vparasmall
4. You may copy and distribute the Library (or a portion or derivative of it, under Section 2) in object code or executable form under the terms of Sections 1 and 2 above provided that you accompany it with the complete corresponding machine-readable source code, which must be distributed under the terms of Sections 1 and 2 above on a medium customarily used for software interchange. 

If distribution of object code is made by offering access to copy from a designated place, then offering equivalent access to copy the source code from the same place satisfies the requirement to distribute the source code, even though third parties are not compelled to copy the source along with the object code. 

\vparasmall
5. A program that contains no derivative of any portion of the Library, but is designed to work with the Library by being compiled or linked with it, is called a "work that uses the Library". Such a work, in isolation, is not a derivative work of the Library, and therefore falls outside the scope of this License. 

However, linking a "work that uses the Library" with the Library creates an executable that is a derivative of the Library (because it contains portions of the Library), rather than a "work that uses the library". The executable is therefore covered by this License. Section 6 states terms for distribution of such executables. 

When a "work that uses the Library" uses material from a header file that is part of the Library, the object code for the work may be a derivative work of the Library even though the source code is not. Whether this is true is especially significant if the work can be linked without the Library, or if the work is itself a library. The threshold for this to be true is not precisely defined by law. 

If such an object file uses only numerical parameters, data structure layouts and accessors, and small macros and small inline functions (ten lines or less in length), then the use of the object file is unrestricted, regardless of whether it is legally a derivative work. (Executables containing this object code plus portions of the Library will still fall under Section 6.) 

Otherwise, if the work is a derivative of the Library, you may distribute the object code for the work under the terms of Section 6. Any executables containing that work also fall under Section 6, whether or not they are linked directly with the Library itself. 

\vparasmall
6. As an exception to the Sections above, you may also compile or link a "work that uses the Library" with the Library to produce a work containing portions of the Library, and distribute that work under terms of your choice, provided that the terms permit modification of the work for the customer's own use and reverse engineering for debugging such modifications. 

You must give prominent notice with each copy of the work that the Library is used in it and that the Library and its use are covered by this License. You must supply a copy of this License. If the work during execution displays copyright notices, you must include the copyright notice for the Library among them, as well as a reference directing the user to the copy of this License. Also, you must do one of these things: 

•a) Accompany the work with the complete corresponding machine-readable source code for the Library including whatever changes were used in the work (which must be distributed under Sections 1 and 2 above); and, if the work is an executable linked with the Library, with the complete machine-readable "work that uses the Library", as object code and/or source code, so that the user can modify the Library and then relink to produce a modified executable containing the modified Library. (It is understood that the user who changes the contents of definitions files in the Library will not necessarily be able to recompile the application to use the modified definitions.) 
•b) Accompany the work with a written offer, valid for at least three years, to give the same user the materials specified in Subsection 6a, above, for a charge no more than the cost of performing this distribution. 
•c) If distribution of the work is made by offering access to copy from a designated place, offer equivalent access to copy the above specified materials from the same place. 
•d) Verify that the user has already received a copy of these materials or that you have already sent this user a copy. 
For an executable, the required form of the "work that uses the Library" must include any data and utility programs needed for reproducing the executable from it. However, as a special exception, the source code distributed need not include anything that is normally distributed (in either source or binary form) with the major components (compiler, kernel, and so on) of the operating system on which the executable runs, unless that component itself accompanies the executable. 

It may happen that this requirement contradicts the license restrictions of other proprietary libraries that do not normally accompany the operating system. Such a contradiction means you cannot use both them and the Library together in an executable that you distribute. 

\vparasmall
7. You may place library facilities that are a work based on the Library side-by-side in a single library together with other library facilities not covered by this License, and distribute such a combined library, provided that the separate distribution of the work based on the Library and of the other library facilities is otherwise permitted, and provided that you do these two things: 

•a) Accompany the combined library with a copy of the same work based on the Library, uncombined with any other library facilities. This must be distributed under the terms of the Sections above. 
•b) Give prominent notice with the combined library of the fact that part of it is a work based on the Library, and explaining where to find the accompanying uncombined form of the same work. 

\vparasmall
8. You may not copy, modify, sublicense, link with, or distribute the Library except as expressly provided under this License. Any attempt otherwise to copy, modify, sublicense, link with, or distribute the Library is void, and will automatically terminate your rights under this License. However, parties who have received copies, or rights, from you under this License will not have their licenses terminated so long as such parties remain in full compliance. 

\vparasmall
9. You are not required to accept this License, since you have not signed it. However, nothing else grants you permission to modify or distribute the Library or its derivative works. These actions are prohibited by law if you do not accept this License. Therefore, by modifying or distributing the Library (or any work based on the Library), you indicate your acceptance of this License to do so, and all its terms and conditions for copying, distributing or modifying the Library or works based on it. 

\vparasmall
10. Each time you redistribute the Library (or any work based on the Library), the recipient automatically receives a license from the original licensor to copy, distribute, link with or modify the Library subject to these terms and conditions. You may not impose any further restrictions on the recipients' exercise of the rights granted herein. You are not responsible for enforcing compliance by third parties to this License. 

\vparasmall
11. If, as a consequence of a court judgment or allegation of patent infringement or for any other reason (not limited to patent issues), conditions are imposed on you (whether by court order, agreement or otherwise) that contradict the conditions of this License, they do not excuse you from the conditions of this License. If you cannot distribute so as to satisfy simultaneously your obligations under this License and any other pertinent obligations, then as a consequence you may not distribute the Library at all. For example, if a patent license would not permit royalty-free redistribution of the Library by all those who receive copies directly or indirectly through you, then the only way you could satisfy both it and this License would be to refrain entirely from distribution of the Library. 

If any portion of this section is held invalid or unenforceable under any particular circumstance, the balance of the section is intended to apply, and the section as a whole is intended to apply in other circumstances. 

It is not the purpose of this section to induce you to infringe any patents or other property right claims or to contest validity of any such claims; this section has the sole purpose of protecting the integrity of the free software distribution system which is implemented by public license practices. Many people have made generous contributions to the wide range of software distributed through that system in reliance on consistent application of that system; it is up to the author/donor to decide if he or she is willing to distribute software through any other system and a licensee cannot impose that choice. 

This section is intended to make thoroughly clear what is believed to be a consequence of the rest of this License. 

\vparasmall
12. If the distribution and/or use of the Library is restricted in certain countries either by patents or by copyrighted interfaces, the original copyright holder who places the Library under this License may add an explicit geographical distribution limitation excluding those countries, so that distribution is permitted only in or among countries not thus excluded. In such case, this License incorporates the limitation as if written in the body of this License. 

\vparasmall
13. The Free Software Foundation may publish revised and/or new versions of the Library General Public License from time to time. Such new versions will be similar in spirit to the present version, but may differ in detail to address new problems or concerns. 

Each version is given a distinguishing version number. If the Library specifies a version number of this License which applies to it and "any later version", you have the option of following the terms and conditions either of that version or of any later version published by the Free Software Foundation. If the Library does not specify a license version number, you may choose any version ever published by the Free Software Foundation. 

\vparasmall
14. If you wish to incorporate parts of the Library into other free programs whose distribution conditions are incompatible with these, write to the author to ask for permission. For software which is copyrighted by the Free Software Foundation, write to the Free Software Foundation; we sometimes make exceptions for this. Our decision will be guided by the two goals of preserving the free status of all derivatives of our free software and of promoting the sharing and reuse of software generally. 

\vparasmall
NO WARRANTY

\vparasmall
15. BECAUSE THE LIBRARY IS LICENSED FREE OF CHARGE, THERE IS NO WARRANTY FOR THE LIBRARY, TO THE EXTENT PERMITTED BY APPLICABLE LAW. EXCEPT WHEN OTHERWISE STATED IN WRITING THE COPYRIGHT HOLDERS AND/OR OTHER PARTIES PROVIDE THE LIBRARY "AS IS" WITHOUT WARRANTY OF ANY KIND, EITHER EXPRESSED OR IMPLIED, INCLUDING, BUT NOT LIMITED TO, THE IMPLIED WARRANTIES OF MERCHANTABILITY AND FITNESS FOR A PARTICULAR PURPOSE. THE ENTIRE RISK AS TO THE QUALITY AND PERFORMANCE OF THE LIBRARY IS WITH YOU. SHOULD THE LIBRARY PROVE DEFECTIVE, YOU ASSUME THE COST OF ALL NECESSARY SERVICING, REPAIR OR CORRECTION. 

\vparasmall
16. IN NO EVENT UNLESS REQUIRED BY APPLICABLE LAW OR AGREED TO IN WRITING WILL ANY COPYRIGHT HOLDER, OR ANY OTHER PARTY WHO MAY MODIFY AND/OR REDISTRIBUTE THE LIBRARY AS PERMITTED ABOVE, BE LIABLE TO YOU FOR DAMAGES, INCLUDING ANY GENERAL, SPECIAL, INCIDENTAL OR CONSEQUENTIAL DAMAGES ARISING OUT OF THE USE OR INABILITY TO USE THE LIBRARY (INCLUDING BUT NOT LIMITED TO LOSS OF DATA OR DATA BEING RENDERED INACCURATE OR LOSSES SUSTAINED BY YOU OR THIRD PARTIES OR A FAILURE OF THE LIBRARY TO OPERATE WITH ANY OTHER SOFTWARE), EVEN IF SUCH HOLDER OR OTHER PARTY HAS BEEN ADVISED OF THE POSSIBILITY OF SUCH DAMAGES. 

END OF TERMS AND CONDITIONS
\vparasmall
How to Apply These Terms to Your New Libraries

\vparasmall
If you develop a new library, and you want it to be of the greatest possible use to the public, we recommend making it free software that everyone can redistribute and change. You can do so by permitting redistribution under these terms (or, alternatively, under the terms of the ordinary General Public License). 

To apply these terms, attach the following notices to the library. It is safest to attach them to the start of each source file to most effectively convey the exclusion of warranty; and each file should have at least the "copyright" line and a pointer to where the full notice is found. 

one line to give the library's name and an idea of what it does.
Copyright (C) year  name of author

This library is free software; you can redistribute it and/or
modify it under the terms of the GNU Library General Public
License as published by the Free Software Foundation; either
version 2 of the License, or (at your option) any later version.

This library is distributed in the hope that it will be useful,
but WITHOUT ANY WARRANTY; without even the implied warranty of
MERCHANTABILITY or FITNESS FOR A PARTICULAR PURPOSE.  See the GNU
Library General Public License for more details.

You should have received a copy of the GNU Library General Public
License along with this library; if not, write to the
Free Software Foundation, Inc., 51 Franklin St, Fifth Floor,
Boston, MA  02110-1301, USA.
Also add information on how to contact you by electronic and paper mail. 

You should also get your employer (if you work as a programmer) or your school, if any, to sign a "copyright disclaimer" for the library, if necessary. Here is a sample; alter the names: 

Yoyodyne, Inc., hereby disclaims all copyright interest in
the library `Frob' (a library for tweaking knobs) written
by James Random Hacker.

signature of Ty Coon, 1 April 1990
Ty Coon, President of Vice
That's all there is to it!






\newpage	
\subsection{GNU Library General Public License, Version 2}
\label{LGPLv2}
\begin{center}
	GNU LIBRARY GENERAL PUBLIC LICENSE
	
	Version 2, June 1991 
\end{center}


\noindent Copyright (C) 1991 Free Software Foundation, Inc.

51 Franklin St, Fifth Floor, Boston, MA  02110-1301, USA


Everyone is permitted to copy and distribute verbatim copies
of this license document, but changing it is not allowed.

[This is the first released version of the library GPL.  It is
numbered 2 because it goes with version 2 of the ordinary GPL.]
Preamble
The licenses for most software are designed to take away your freedom to share and change it. By contrast, the GNU General Public Licenses are intended to guarantee your freedom to share and change free software--to make sure the software is free for all its users. 

This license, the Library General Public License, applies to some specially designated Free Software Foundation software, and to any other libraries whose authors decide to use it. You can use it for your libraries, too. 

When we speak of free software, we are referring to freedom, not price. Our General Public Licenses are designed to make sure that you have the freedom to distribute copies of free software (and charge for this service if you wish), that you receive source code or can get it if you want it, that you can change the software or use pieces of it in new free programs; and that you know you can do these things. 

To protect your rights, we need to make restrictions that forbid anyone to deny you these rights or to ask you to surrender the rights. These restrictions translate to certain responsibilities for you if you distribute copies of the library, or if you modify it. 

For example, if you distribute copies of the library, whether gratis or for a fee, you must give the recipients all the rights that we gave you. You must make sure that they, too, receive or can get the source code. If you link a program with the library, you must provide complete object files to the recipients so that they can relink them with the library, after making changes to the library and recompiling it. And you must show them these terms so they know their rights. 

Our method of protecting your rights has two steps: (1) copyright the library, and (2) offer you this license which gives you legal permission to copy, distribute and/or modify the library. 

Also, for each distributor's protection, we want to make certain that everyone understands that there is no warranty for this free library. If the library is modified by someone else and passed on, we want its recipients to know that what they have is not the original version, so that any problems introduced by others will not reflect on the original authors' reputations. 

Finally, any free program is threatened constantly by software patents. We wish to avoid the danger that companies distributing free software will individually obtain patent licenses, thus in effect transforming the program into proprietary software. To prevent this, we have made it clear that any patent must be licensed for everyone's free use or not licensed at all. 

Most GNU software, including some libraries, is covered by the ordinary GNU General Public License, which was designed for utility programs. This license, the GNU Library General Public License, applies to certain designated libraries. This license is quite different from the ordinary one; be sure to read it in full, and don't assume that anything in it is the same as in the ordinary license. 

The reason we have a separate public license for some libraries is that they blur the distinction we usually make between modifying or adding to a program and simply using it. Linking a program with a library, without changing the library, is in some sense simply using the library, and is analogous to running a utility program or application program. However, in a textual and legal sense, the linked executable is a combined work, a derivative of the original library, and the ordinary General Public License treats it as such. 

Because of this blurred distinction, using the ordinary General Public License for libraries did not effectively promote software sharing, because most developers did not use the libraries. We concluded that weaker conditions might promote sharing better. 

However, unrestricted linking of non-free programs would deprive the users of those programs of all benefit from the free status of the libraries themselves. This Library General Public License is intended to permit developers of non-free programs to use free libraries, while preserving your freedom as a user of such programs to change the free libraries that are incorporated in them. (We have not seen how to achieve this as regards changes in header files, but we have achieved it as regards changes in the actual functions of the Library.) The hope is that this will lead to faster development of free libraries. 

The precise terms and conditions for copying, distribution and modification follow. Pay close attention to the difference between a "work based on the library" and a "work that uses the library". The former contains code derived from the library, while the latter only works together with the library. 

Note that it is possible for a library to be covered by the ordinary General Public License rather than by this special one. 

\vparasmall
TERMS AND CONDITIONS FOR COPYING, DISTRIBUTION AND MODIFICATION

\vparasmall
0. This License Agreement applies to any software library which contains a notice placed by the copyright holder or other authorized party saying it may be distributed under the terms of this Library General Public License (also called "this License"). Each licensee is addressed as "you". 

A "library" means a collection of software functions and/or data prepared so as to be conveniently linked with application programs (which use some of those functions and data) to form executables. 

The "Library", below, refers to any such software library or work which has been distributed under these terms. A "work based on the Library" means either the Library or any derivative work under copyright law: that is to say, a work containing the Library or a portion of it, either verbatim or with modifications and/or translated straightforwardly into another language. (Hereinafter, translation is included without limitation in the term "modification".) 

"Source code" for a work means the preferred form of the work for making modifications to it. For a library, complete source code means all the source code for all modules it contains, plus any associated interface definition files, plus the scripts used to control compilation and installation of the library. 

Activities other than copying, distribution and modification are not covered by this License; they are outside its scope. The act of running a program using the Library is not restricted, and output from such a program is covered only if its contents constitute a work based on the Library (independent of the use of the Library in a tool for writing it). Whether that is true depends on what the Library does and what the program that uses the Library does. 

\vparasmall
1. You may copy and distribute verbatim copies of the Library's complete source code as you receive it, in any medium, provided that you conspicuously and appropriately publish on each copy an appropriate copyright notice and disclaimer of warranty; keep intact all the notices that refer to this License and to the absence of any warranty; and distribute a copy of this License along with the Library. 

You may charge a fee for the physical act of transferring a copy, and you may at your option offer warranty protection in exchange for a fee. 

\vparasmall
2. You may modify your copy or copies of the Library or any portion of it, thus forming a work based on the Library, and copy and distribute such modifications or work under the terms of Section 1 above, provided that you also meet all of these conditions: 

•a) The modified work must itself be a software library. 
•b) You must cause the files modified to carry prominent notices stating that you changed the files and the date of any change. 
•c) You must cause the whole of the work to be licensed at no charge to all third parties under the terms of this License. 
•d) If a facility in the modified Library refers to a function or a table of data to be supplied by an application program that uses the facility, other than as an argument passed when the facility is invoked, then you must make a good faith effort to ensure that, in the event an application does not supply such function or table, the facility still operates, and performs whatever part of its purpose remains meaningful. 
(For example, a function in a library to compute square roots has a purpose that is entirely well-defined independent of the application. Therefore, Subsection 2d requires that any application-supplied function or table used by this function must be optional: if the application does not supply it, the square root function must still compute square roots.)

These requirements apply to the modified work as a whole. If identifiable sections of that work are not derived from the Library, and can be reasonably considered independent and separate works in themselves, then this License, and its terms, do not apply to those sections when you distribute them as separate works. But when you distribute the same sections as part of a whole which is a work based on the Library, the distribution of the whole must be on the terms of this License, whose permissions for other licensees extend to the entire whole, and thus to each and every part regardless of who wrote it. 

Thus, it is not the intent of this section to claim rights or contest your rights to work written entirely by you; rather, the intent is to exercise the right to control the distribution of derivative or collective works based on the Library. 

In addition, mere aggregation of another work not based on the Library with the Library (or with a work based on the Library) on a volume of a storage or distribution medium does not bring the other work under the scope of this License. 

\vparasmall
3. You may opt to apply the terms of the ordinary GNU General Public License instead of this License to a given copy of the Library. To do this, you must alter all the notices that refer to this License, so that they refer to the ordinary GNU General Public License, version 2, instead of to this License. (If a newer version than version 2 of the ordinary GNU General Public License has appeared, then you can specify that version instead if you wish.) Do not make any other change in these notices. 

Once this change is made in a given copy, it is irreversible for that copy, so the ordinary GNU General Public License applies to all subsequent copies and derivative works made from that copy. 

This option is useful when you wish to copy part of the code of the Library into a program that is not a library. 

\vparasmall
4. You may copy and distribute the Library (or a portion or derivative of it, under Section 2) in object code or executable form under the terms of Sections 1 and 2 above provided that you accompany it with the complete corresponding machine-readable source code, which must be distributed under the terms of Sections 1 and 2 above on a medium customarily used for software interchange. 

If distribution of object code is made by offering access to copy from a designated place, then offering equivalent access to copy the source code from the same place satisfies the requirement to distribute the source code, even though third parties are not compelled to copy the source along with the object code. 

\vparasmall
5. A program that contains no derivative of any portion of the Library, but is designed to work with the Library by being compiled or linked with it, is called a "work that uses the Library". Such a work, in isolation, is not a derivative work of the Library, and therefore falls outside the scope of this License. 

However, linking a "work that uses the Library" with the Library creates an executable that is a derivative of the Library (because it contains portions of the Library), rather than a "work that uses the library". The executable is therefore covered by this License. Section 6 states terms for distribution of such executables. 

When a "work that uses the Library" uses material from a header file that is part of the Library, the object code for the work may be a derivative work of the Library even though the source code is not. Whether this is true is especially significant if the work can be linked without the Library, or if the work is itself a library. The threshold for this to be true is not precisely defined by law. 

If such an object file uses only numerical parameters, data structure layouts and accessors, and small macros and small inline functions (ten lines or less in length), then the use of the object file is unrestricted, regardless of whether it is legally a derivative work. (Executables containing this object code plus portions of the Library will still fall under Section 6.) 

Otherwise, if the work is a derivative of the Library, you may distribute the object code for the work under the terms of Section 6. Any executables containing that work also fall under Section 6, whether or not they are linked directly with the Library itself. 

\vparasmall
6. As an exception to the Sections above, you may also compile or link a "work that uses the Library" with the Library to produce a work containing portions of the Library, and distribute that work under terms of your choice, provided that the terms permit modification of the work for the customer's own use and reverse engineering for debugging such modifications. 

You must give prominent notice with each copy of the work that the Library is used in it and that the Library and its use are covered by this License. You must supply a copy of this License. If the work during execution displays copyright notices, you must include the copyright notice for the Library among them, as well as a reference directing the user to the copy of this License. Also, you must do one of these things: 

•a) Accompany the work with the complete corresponding machine-readable source code for the Library including whatever changes were used in the work (which must be distributed under Sections 1 and 2 above); and, if the work is an executable linked with the Library, with the complete machine-readable "work that uses the Library", as object code and/or source code, so that the user can modify the Library and then relink to produce a modified executable containing the modified Library. (It is understood that the user who changes the contents of definitions files in the Library will not necessarily be able to recompile the application to use the modified definitions.) 
•b) Accompany the work with a written offer, valid for at least three years, to give the same user the materials specified in Subsection 6a, above, for a charge no more than the cost of performing this distribution. 
•c) If distribution of the work is made by offering access to copy from a designated place, offer equivalent access to copy the above specified materials from the same place. 
•d) Verify that the user has already received a copy of these materials or that you have already sent this user a copy. 
For an executable, the required form of the "work that uses the Library" must include any data and utility programs needed for reproducing the executable from it. However, as a special exception, the source code distributed need not include anything that is normally distributed (in either source or binary form) with the major components (compiler, kernel, and so on) of the operating system on which the executable runs, unless that component itself accompanies the executable. 

It may happen that this requirement contradicts the license restrictions of other proprietary libraries that do not normally accompany the operating system. Such a contradiction means you cannot use both them and the Library together in an executable that you distribute. 

\vparasmall
7. You may place library facilities that are a work based on the Library side-by-side in a single library together with other library facilities not covered by this License, and distribute such a combined library, provided that the separate distribution of the work based on the Library and of the other library facilities is otherwise permitted, and provided that you do these two things: 

•a) Accompany the combined library with a copy of the same work based on the Library, uncombined with any other library facilities. This must be distributed under the terms of the Sections above. 
•b) Give prominent notice with the combined library of the fact that part of it is a work based on the Library, and explaining where to find the accompanying uncombined form of the same work. 

\vparasmall
8. You may not copy, modify, sublicense, link with, or distribute the Library except as expressly provided under this License. Any attempt otherwise to copy, modify, sublicense, link with, or distribute the Library is void, and will automatically terminate your rights under this License. However, parties who have received copies, or rights, from you under this License will not have their licenses terminated so long as such parties remain in full compliance. 

\vparasmall
9. You are not required to accept this License, since you have not signed it. However, nothing else grants you permission to modify or distribute the Library or its derivative works. These actions are prohibited by law if you do not accept this License. Therefore, by modifying or distributing the Library (or any work based on the Library), you indicate your acceptance of this License to do so, and all its terms and conditions for copying, distributing or modifying the Library or works based on it. 

\vparasmall
10. Each time you redistribute the Library (or any work based on the Library), the recipient automatically receives a license from the original licensor to copy, distribute, link with or modify the Library subject to these terms and conditions. You may not impose any further restrictions on the recipients' exercise of the rights granted herein. You are not responsible for enforcing compliance by third parties to this License. 

\vparasmall
11. If, as a consequence of a court judgment or allegation of patent infringement or for any other reason (not limited to patent issues), conditions are imposed on you (whether by court order, agreement or otherwise) that contradict the conditions of this License, they do not excuse you from the conditions of this License. If you cannot distribute so as to satisfy simultaneously your obligations under this License and any other pertinent obligations, then as a consequence you may not distribute the Library at all. For example, if a patent license would not permit royalty-free redistribution of the Library by all those who receive copies directly or indirectly through you, then the only way you could satisfy both it and this License would be to refrain entirely from distribution of the Library. 

If any portion of this section is held invalid or unenforceable under any particular circumstance, the balance of the section is intended to apply, and the section as a whole is intended to apply in other circumstances. 

It is not the purpose of this section to induce you to infringe any patents or other property right claims or to contest validity of any such claims; this section has the sole purpose of protecting the integrity of the free software distribution system which is implemented by public license practices. Many people have made generous contributions to the wide range of software distributed through that system in reliance on consistent application of that system; it is up to the author/donor to decide if he or she is willing to distribute software through any other system and a licensee cannot impose that choice. 

This section is intended to make thoroughly clear what is believed to be a consequence of the rest of this License. 

\vparasmall
12. If the distribution and/or use of the Library is restricted in certain countries either by patents or by copyrighted interfaces, the original copyright holder who places the Library under this License may add an explicit geographical distribution limitation excluding those countries, so that distribution is permitted only in or among countries not thus excluded. In such case, this License incorporates the limitation as if written in the body of this License. 

\vparasmall
13. The Free Software Foundation may publish revised and/or new versions of the Library General Public License from time to time. Such new versions will be similar in spirit to the present version, but may differ in detail to address new problems or concerns. 

Each version is given a distinguishing version number. If the Library specifies a version number of this License which applies to it and "any later version", you have the option of following the terms and conditions either of that version or of any later version published by the Free Software Foundation. If the Library does not specify a license version number, you may choose any version ever published by the Free Software Foundation. 

\vparasmall
14. If you wish to incorporate parts of the Library into other free programs whose distribution conditions are incompatible with these, write to the author to ask for permission. For software which is copyrighted by the Free Software Foundation, write to the Free Software Foundation; we sometimes make exceptions for this. Our decision will be guided by the two goals of preserving the free status of all derivatives of our free software and of promoting the sharing and reuse of software generally. 

\vparasmall
NO WARRANTY

\vparasmall
15. BECAUSE THE LIBRARY IS LICENSED FREE OF CHARGE, THERE IS NO WARRANTY FOR THE LIBRARY, TO THE EXTENT PERMITTED BY APPLICABLE LAW. EXCEPT WHEN OTHERWISE STATED IN WRITING THE COPYRIGHT HOLDERS AND/OR OTHER PARTIES PROVIDE THE LIBRARY "AS IS" WITHOUT WARRANTY OF ANY KIND, EITHER EXPRESSED OR IMPLIED, INCLUDING, BUT NOT LIMITED TO, THE IMPLIED WARRANTIES OF MERCHANTABILITY AND FITNESS FOR A PARTICULAR PURPOSE. THE ENTIRE RISK AS TO THE QUALITY AND PERFORMANCE OF THE LIBRARY IS WITH YOU. SHOULD THE LIBRARY PROVE DEFECTIVE, YOU ASSUME THE COST OF ALL NECESSARY SERVICING, REPAIR OR CORRECTION. 

\vparasmall
16. IN NO EVENT UNLESS REQUIRED BY APPLICABLE LAW OR AGREED TO IN WRITING WILL ANY COPYRIGHT HOLDER, OR ANY OTHER PARTY WHO MAY MODIFY AND/OR REDISTRIBUTE THE LIBRARY AS PERMITTED ABOVE, BE LIABLE TO YOU FOR DAMAGES, INCLUDING ANY GENERAL, SPECIAL, INCIDENTAL OR CONSEQUENTIAL DAMAGES ARISING OUT OF THE USE OR INABILITY TO USE THE LIBRARY (INCLUDING BUT NOT LIMITED TO LOSS OF DATA OR DATA BEING RENDERED INACCURATE OR LOSSES SUSTAINED BY YOU OR THIRD PARTIES OR A FAILURE OF THE LIBRARY TO OPERATE WITH ANY OTHER SOFTWARE), EVEN IF SUCH HOLDER OR OTHER PARTY HAS BEEN ADVISED OF THE POSSIBILITY OF SUCH DAMAGES. 

END OF TERMS AND CONDITIONS
\vparasmall
How to Apply These Terms to Your New Libraries

\vparasmall
If you develop a new library, and you want it to be of the greatest possible use to the public, we recommend making it free software that everyone can redistribute and change. You can do so by permitting redistribution under these terms (or, alternatively, under the terms of the ordinary General Public License). 

To apply these terms, attach the following notices to the library. It is safest to attach them to the start of each source file to most effectively convey the exclusion of warranty; and each file should have at least the "copyright" line and a pointer to where the full notice is found. 

one line to give the library's name and an idea of what it does.
Copyright (C) year  name of author

This library is free software; you can redistribute it and/or
modify it under the terms of the GNU Library General Public
License as published by the Free Software Foundation; either
version 2 of the License, or (at your option) any later version.

This library is distributed in the hope that it will be useful,
but WITHOUT ANY WARRANTY; without even the implied warranty of
MERCHANTABILITY or FITNESS FOR A PARTICULAR PURPOSE.  See the GNU
Library General Public License for more details.

You should have received a copy of the GNU Library General Public
License along with this library; if not, write to the
Free Software Foundation, Inc., 51 Franklin St, Fifth Floor,
Boston, MA  02110-1301, USA.
Also add information on how to contact you by electronic and paper mail. 

You should also get your employer (if you work as a programmer) or your school, if any, to sign a "copyright disclaimer" for the library, if necessary. Here is a sample; alter the names: 

Yoyodyne, Inc., hereby disclaims all copyright interest in
the library `Frob' (a library for tweaking knobs) written
by James Random Hacker.

signature of Ty Coon, 1 April 1990
Ty Coon, President of Vice
That's all there is to it!



















\newpage
\subsection{GNU Lesser General Public License, Version 3}
\label{LGPLv3}
\begin{center}
	GNU LESSER GENERAL PUBLIC LICENSE
	
	Version 3, 29 June 2007
\end{center}

\noindent Copyright (C) 2007 Free Software Foundation, Inc.  $<$\href{http://fsf.org/}{http://fsf.org/}$>$ \\

\noindent Everyone is permitted to copy and distribute verbatim copies
of this license document, but changing it is not allowed. \\


This version of the GNU Lesser General Public License incorporates
the terms and conditions of version 3 of the GNU General Public
License, supplemented by the additional permissions listed below. \\

0. Additional Definitions.

As used herein, "this License" refers to version 3 of the GNU Lesser
General Public License, and the "GNU GPL" refers to version 3 of the GNU
General Public License.

"The Library" refers to a covered work governed by this License,
other than an Application or a Combined Work as defined below.

An "Application" is any work that makes use of an interface provided
by the Library, but which is not otherwise based on the Library.
Defining a subclass of a class defined by the Library is deemed a mode
of using an interface provided by the Library.

A "Combined Work" is a work produced by combining or linking an
Application with the Library.  The particular version of the Library
with which the Combined Work was made is also called the "Linked
Version".

The "Minimal Corresponding Source" for a Combined Work means the
Corresponding Source for the Combined Work, excluding any source code
for portions of the Combined Work that, considered in isolation, are
based on the Application, and not on the Linked Version.

The "Corresponding Application Code" for a Combined Work means the
object code and/or source code for the Application, including any data
and utility programs needed for reproducing the Combined Work from the
Application, but excluding the System Libraries of the Combined Work. \\

1. Exception to Section 3 of the GNU GPL.

You may convey a covered work under sections 3 and 4 of this License
without being bound by section 3 of the GNU GPL. \\

2. Conveying Modified Versions.

If you modify a copy of the Library, and, in your modifications, a
facility refers to a function or data to be supplied by an Application
that uses the facility (other than as an argument passed when the
facility is invoked), then you may convey a copy of the modified
version:

a) under this License, provided that you make a good faith effort to
ensure that, in the event an Application does not supply the
function or data, the facility still operates, and performs
whatever part of its purpose remains meaningful, or

b) under the GNU GPL, with none of the additional permissions of
this License applicable to that copy. \\

3. Object Code Incorporating Material from Library Header Files.

The object code form of an Application may incorporate material from
a header file that is part of the Library.  You may convey such object
code under terms of your choice, provided that, if the incorporated
material is not limited to numerical parameters, data structure
layouts and accessors, or small macros, inline functions and templates
(ten or fewer lines in length), you do both of the following:

a) Give prominent notice with each copy of the object code that the
Library is used in it and that the Library and its use are
covered by this License.

b) Accompany the object code with a copy of the GNU GPL and this license
document. \\

4. Combined Works.

You may convey a Combined Work under terms of your choice that,
taken together, effectively do not restrict modification of the
portions of the Library contained in the Combined Work and reverse
engineering for debugging such modifications, if you also do each of
the following:

a) Give prominent notice with each copy of the Combined Work that
the Library is used in it and that the Library and its use are
covered by this License.

b) Accompany the Combined Work with a copy of the GNU GPL and this license
document.

c) For a Combined Work that displays copyright notices during
execution, include the copyright notice for the Library among
these notices, as well as a reference directing the user to the
copies of the GNU GPL and this license document.

d) Do one of the following:

0) Convey the Minimal Corresponding Source under the terms of this
License, and the Corresponding Application Code in a form
suitable for, and under terms that permit, the user to
recombine or relink the Application with a modified version of
the Linked Version to produce a modified Combined Work, in the
manner specified by section 6 of the GNU GPL for conveying
Corresponding Source.

1) Use a suitable shared library mechanism for linking with the
Library.  A suitable mechanism is one that (a) uses at run time
a copy of the Library already present on the user's computer
system, and (b) will operate properly with a modified version
of the Library that is interface-compatible with the Linked
Version.

e) Provide Installation Information, but only if you would otherwise
be required to provide such information under section 6 of the
GNU GPL, and only to the extent that such information is
necessary to install and execute a modified version of the
Combined Work produced by recombining or relinking the
Application with a modified version of the Linked Version. (If
you use option 4d0, the Installation Information must accompany
the Minimal Corresponding Source and Corresponding Application
Code. If you use option 4d1, you must provide the Installation
Information in the manner specified by section 6 of the GNU GPL
for conveying Corresponding Source.) \\

5. Combined Libraries.

You may place library facilities that are a work based on the
Library side by side in a single library together with other library
facilities that are not Applications and are not covered by this
License, and convey such a combined library under terms of your
choice, if you do both of the following:

a) Accompany the combined library with a copy of the same work based
on the Library, uncombined with any other library facilities,
conveyed under the terms of this License.

b) Give prominent notice with the combined library that part of it
is a work based on the Library, and explaining where to find the
accompanying uncombined form of the same work. \\

6. Revised Versions of the GNU Lesser General Public License.

The Free Software Foundation may publish revised and/or new versions
of the GNU Lesser General Public License from time to time. Such new
versions will be similar in spirit to the present version, but may
differ in detail to address new problems or concerns.

Each version is given a distinguishing version number. If the
Library as you received it specifies that a certain numbered version
of the GNU Lesser General Public License "or any later version"
applies to it, you have the option of following the terms and
conditions either of that published version or of any later version
published by the Free Software Foundation. If the Library as you
received it does not specify a version number of the GNU Lesser
General Public License, you may choose any version of the GNU Lesser
General Public License ever published by the Free Software Foundation.

If the Library as you received it specifies that a proxy can decide
whether future versions of the GNU Lesser General Public License shall
apply, that proxy's public statement of acceptance of any version is
permanent authorization for you to choose that version for the
Library.


\newpage 



\subsection{GNU General Public License, Version 3}
\label{GPLv3}
\begin{center}
	GNU GENERAL PUBLIC LICENSE
	
	Version 3, 29 June 2007
\end{center}

\noindent Copyright (C) 2007 Free Software Foundation, Inc.  $<$\href{http://fsf.org/}{http://fsf.org/}$>$ \\

\noindent Everyone is permitted to copy and distribute verbatim copies
of this license document, but changing it is not allowed. \\



Preamble

The GNU General Public License is a free, copyleft license for
software and other kinds of works.

The licenses for most software and other practical works are designed
to take away your freedom to share and change the works.  By contrast,
the GNU General Public License is intended to guarantee your freedom to
share and change all versions of a program--to make sure it remains free
software for all its users.  We, the Free Software Foundation, use the
GNU General Public License for most of our software; it applies also to
any other work released this way by its authors.  You can apply it to
your programs, too.

When we speak of free software, we are referring to freedom, not
price.  Our General Public Licenses are designed to make sure that you
have the freedom to distribute copies of free software (and charge for
them if you wish), that you receive source code or can get it if you
want it, that you can change the software or use pieces of it in new
free programs, and that you know you can do these things.

To protect your rights, we need to prevent others from denying you
these rights or asking you to surrender the rights.  Therefore, you have
certain responsibilities if you distribute copies of the software, or if
you modify it: responsibilities to respect the freedom of others.

For example, if you distribute copies of such a program, whether
gratis or for a fee, you must pass on to the recipients the same
freedoms that you received.  You must make sure that they, too, receive
or can get the source code.  And you must show them these terms so they
know their rights.

Developers that use the GNU GPL protect your rights with two steps:
(1) assert copyright on the software, and (2) offer you this License
giving you legal permission to copy, distribute and/or modify it.

For the developers' and authors' protection, the GPL clearly explains
that there is no warranty for this free software.  For both users' and
authors' sake, the GPL requires that modified versions be marked as
changed, so that their problems will not be attributed erroneously to
authors of previous versions.

Some devices are designed to deny users access to install or run
modified versions of the software inside them, although the manufacturer
can do so.  This is fundamentally incompatible with the aim of
protecting users' freedom to change the software.  The systematic
pattern of such abuse occurs in the area of products for individuals to
use, which is precisely where it is most unacceptable.  Therefore, we
have designed this version of the GPL to prohibit the practice for those
products.  If such problems arise substantially in other domains, we
stand ready to extend this provision to those domains in future versions
of the GPL, as needed to protect the freedom of users.

Finally, every program is threatened constantly by software patents.
States should not allow patents to restrict development and use of
software on general-purpose computers, but in those that do, we wish to
avoid the special danger that patents applied to a free program could
make it effectively proprietary.  To prevent this, the GPL assures that
patents cannot be used to render the program non-free.

The precise terms and conditions for copying, distribution and
modification follow.

\vparasmall
TERMS AND CONDITIONS

\vparasmall
0. Definitions.

"This License" refers to version 3 of the GNU General Public License.

"Copyright" also means copyright-like laws that apply to other kinds of
works, such as semiconductor masks.

"The Program" refers to any copyrightable work licensed under this
License.  Each licensee is addressed as "you".  "Licensees" and
"recipients" may be individuals or organizations.

To "modify" a work means to copy from or adapt all or part of the work
in a fashion requiring copyright permission, other than the making of an
exact copy.  The resulting work is called a "modified version" of the
earlier work or a work "based on" the earlier work.

A "covered work" means either the unmodified Program or a work based
on the Program.

To "propagate" a work means to do anything with it that, without
permission, would make you directly or secondarily liable for
infringement under applicable copyright law, except executing it on a
computer or modifying a private copy.  Propagation includes copying,
distribution (with or without modification), making available to the
public, and in some countries other activities as well.

To "convey" a work means any kind of propagation that enables other
parties to make or receive copies.  Mere interaction with a user through
a computer network, with no transfer of a copy, is not conveying.

An interactive user interface displays "Appropriate Legal Notices"
to the extent that it includes a convenient and prominently visible
feature that (1) displays an appropriate copyright notice, and (2)
tells the user that there is no warranty for the work (except to the
extent that warranties are provided), that licensees may convey the
work under this License, and how to view a copy of this License.  If
the interface presents a list of user commands or options, such as a
menu, a prominent item in the list meets this criterion.

\vparasmall
1. Source Code.

The "source code" for a work means the preferred form of the work
for making modifications to it.  "Object code" means any non-source
form of a work.

A "Standard Interface" means an interface that either is an official
standard defined by a recognized standards body, or, in the case of
interfaces specified for a particular programming language, one that
is widely used among developers working in that language.

The "System Libraries" of an executable work include anything, other
than the work as a whole, that (a) is included in the normal form of
packaging a Major Component, but which is not part of that Major
Component, and (b) serves only to enable use of the work with that
Major Component, or to implement a Standard Interface for which an
implementation is available to the public in source code form.  A
"Major Component", in this context, means a major essential component
(kernel, window system, and so on) of the specific operating system
(if any) on which the executable work runs, or a compiler used to
produce the work, or an object code interpreter used to run it.

The "Corresponding Source" for a work in object code form means all
the source code needed to generate, install, and (for an executable
work) run the object code and to modify the work, including scripts to
control those activities.  However, it does not include the work's
System Libraries, or general-purpose tools or generally available free
programs which are used unmodified in performing those activities but
which are not part of the work.  For example, Corresponding Source
includes interface definition files associated with source files for
the work, and the source code for shared libraries and dynamically
linked subprograms that the work is specifically designed to require,
such as by intimate data communication or control flow between those
subprograms and other parts of the work.

The Corresponding Source need not include anything that users
can regenerate automatically from other parts of the Corresponding
Source.

The Corresponding Source for a work in source code form is that
same work.

\vparasmall
2. Basic Permissions.

All rights granted under this License are granted for the term of
copyright on the Program, and are irrevocable provided the stated
conditions are met.  This License explicitly affirms your unlimited
permission to run the unmodified Program.  The output from running a
covered work is covered by this License only if the output, given its
content, constitutes a covered work.  This License acknowledges your
rights of fair use or other equivalent, as provided by copyright law.

You may make, run and propagate covered works that you do not
convey, without conditions so long as your license otherwise remains
in force.  You may convey covered works to others for the sole purpose
of having them make modifications exclusively for you, or provide you
with facilities for running those works, provided that you comply with
the terms of this License in conveying all material for which you do
not control copyright.  Those thus making or running the covered works
for you must do so exclusively on your behalf, under your direction
and control, on terms that prohibit them from making any copies of
your copyrighted material outside their relationship with you.

Conveying under any other circumstances is permitted solely under
the conditions stated below.  Sublicensing is not allowed; section 10
makes it unnecessary.

\vparasmall
3. Protecting Users' Legal Rights From Anti-Circumvention Law.

No covered work shall be deemed part of an effective technological
measure under any applicable law fulfilling obligations under article
11 of the WIPO copyright treaty adopted on 20 December 1996, or
similar laws prohibiting or restricting circumvention of such
measures.

When you convey a covered work, you waive any legal power to forbid
circumvention of technological measures to the extent such circumvention
is effected by exercising rights under this License with respect to
the covered work, and you disclaim any intention to limit operation or
modification of the work as a means of enforcing, against the work's
users, your or third parties' legal rights to forbid circumvention of
technological measures.

\vparasmall
4. Conveying Verbatim Copies.

You may convey verbatim copies of the Program's source code as you
receive it, in any medium, provided that you conspicuously and
appropriately publish on each copy an appropriate copyright notice;
keep intact all notices stating that this License and any
non-permissive terms added in accord with section 7 apply to the code;
keep intact all notices of the absence of any warranty; and give all
recipients a copy of this License along with the Program.

You may charge any price or no price for each copy that you convey,
and you may offer support or warranty protection for a fee.

\vparasmall
5. Conveying Modified Source Versions.

You may convey a work based on the Program, or the modifications to
produce it from the Program, in the form of source code under the
terms of section 4, provided that you also meet all of these conditions:

a) The work must carry prominent notices stating that you modified
it, and giving a relevant date.

b) The work must carry prominent notices stating that it is
released under this License and any conditions added under section
7.  This requirement modifies the requirement in section 4 to
"keep intact all notices".

c) You must license the entire work, as a whole, under this
License to anyone who comes into possession of a copy.  This
License will therefore apply, along with any applicable section 7
additional terms, to the whole of the work, and all its parts,
regardless of how they are packaged.  This License gives no
permission to license the work in any other way, but it does not
invalidate such permission if you have separately received it.

d) If the work has interactive user interfaces, each must display
Appropriate Legal Notices; however, if the Program has interactive
interfaces that do not display Appropriate Legal Notices, your
work need not make them do so.

A compilation of a covered work with other separate and independent
works, which are not by their nature extensions of the covered work,
and which are not combined with it such as to form a larger program,
in or on a volume of a storage or distribution medium, is called an
"aggregate" if the compilation and its resulting copyright are not
used to limit the access or legal rights of the compilation's users
beyond what the individual works permit.  Inclusion of a covered work
in an aggregate does not cause this License to apply to the other
parts of the aggregate.

\vparasmall
6. Conveying Non-Source Forms.

You may convey a covered work in object code form under the terms
of sections 4 and 5, provided that you also convey the
machine-readable Corresponding Source under the terms of this License,
in one of these ways:

a) Convey the object code in, or embodied in, a physical product
(including a physical distribution medium), accompanied by the
Corresponding Source fixed on a durable physical medium
customarily used for software interchange.

b) Convey the object code in, or embodied in, a physical product
(including a physical distribution medium), accompanied by a
written offer, valid for at least three years and valid for as
long as you offer spare parts or customer support for that product
model, to give anyone who possesses the object code either (1) a
copy of the Corresponding Source for all the software in the
product that is covered by this License, on a durable physical
medium customarily used for software interchange, for a price no
more than your reasonable cost of physically performing this
conveying of source, or (2) access to copy the
Corresponding Source from a network server at no charge.

c) Convey individual copies of the object code with a copy of the
written offer to provide the Corresponding Source.  This
alternative is allowed only occasionally and noncommercially, and
only if you received the object code with such an offer, in accord
with subsection 6b.

d) Convey the object code by offering access from a designated
place (gratis or for a charge), and offer equivalent access to the
Corresponding Source in the same way through the same place at no
further charge.  You need not require recipients to copy the
Corresponding Source along with the object code.  If the place to
copy the object code is a network server, the Corresponding Source
may be on a different server (operated by you or a third party)
that supports equivalent copying facilities, provided you maintain
clear directions next to the object code saying where to find the
Corresponding Source.  Regardless of what server hosts the
Corresponding Source, you remain obligated to ensure that it is
available for as long as needed to satisfy these requirements.

e) Convey the object code using peer-to-peer transmission, provided
you inform other peers where the object code and Corresponding
Source of the work are being offered to the general public at no
charge under subsection 6d.

A separable portion of the object code, whose source code is excluded
from the Corresponding Source as a System Library, need not be
included in conveying the object code work.

A "User Product" is either (1) a "consumer product", which means any
tangible personal property which is normally used for personal, family,
or household purposes, or (2) anything designed or sold for incorporation
into a dwelling.  In determining whether a product is a consumer product,
doubtful cases shall be resolved in favor of coverage.  For a particular
product received by a particular user, "normally used" refers to a
typical or common use of that class of product, regardless of the status
of the particular user or of the way in which the particular user
actually uses, or expects or is expected to use, the product.  A product
is a consumer product regardless of whether the product has substantial
commercial, industrial or non-consumer uses, unless such uses represent
the only significant mode of use of the product.

"Installation Information" for a User Product means any methods,
procedures, authorization keys, or other information required to install
and execute modified versions of a covered work in that User Product from
a modified version of its Corresponding Source.  The information must
suffice to ensure that the continued functioning of the modified object
code is in no case prevented or interfered with solely because
modification has been made.

If you convey an object code work under this section in, or with, or
specifically for use in, a User Product, and the conveying occurs as
part of a transaction in which the right of possession and use of the
User Product is transferred to the recipient in perpetuity or for a
fixed term (regardless of how the transaction is characterized), the
Corresponding Source conveyed under this section must be accompanied
by the Installation Information.  But this requirement does not apply
if neither you nor any third party retains the ability to install
modified object code on the User Product (for example, the work has
been installed in ROM).

The requirement to provide Installation Information does not include a
requirement to continue to provide support service, warranty, or updates
for a work that has been modified or installed by the recipient, or for
the User Product in which it has been modified or installed.  Access to a
network may be denied when the modification itself materially and
adversely affects the operation of the network or violates the rules and
protocols for communication across the network.

Corresponding Source conveyed, and Installation Information provided,
in accord with this section must be in a format that is publicly
documented (and with an implementation available to the public in
source code form), and must require no special password or key for
unpacking, reading or copying.

\vparasmall
7. Additional Terms.

"Additional permissions" are terms that supplement the terms of this
License by making exceptions from one or more of its conditions.
Additional permissions that are applicable to the entire Program shall
be treated as though they were included in this License, to the extent
that they are valid under applicable law.  If additional permissions
apply only to part of the Program, that part may be used separately
under those permissions, but the entire Program remains governed by
this License without regard to the additional permissions.

When you convey a copy of a covered work, you may at your option
remove any additional permissions from that copy, or from any part of
it.  (Additional permissions may be written to require their own
removal in certain cases when you modify the work.)  You may place
additional permissions on material, added by you to a covered work,
for which you have or can give appropriate copyright permission.

Notwithstanding any other provision of this License, for material you
add to a covered work, you may (if authorized by the copyright holders of
that material) supplement the terms of this License with terms:

a) Disclaiming warranty or limiting liability differently from the
terms of sections 15 and 16 of this License; or

b) Requiring preservation of specified reasonable legal notices or
author attributions in that material or in the Appropriate Legal
Notices displayed by works containing it; or

c) Prohibiting misrepresentation of the origin of that material, or
requiring that modified versions of such material be marked in
reasonable ways as different from the original version; or

d) Limiting the use for publicity purposes of names of licensors or
authors of the material; or

e) Declining to grant rights under trademark law for use of some
trade names, trademarks, or service marks; or

f) Requiring indemnification of licensors and authors of that
material by anyone who conveys the material (or modified versions of
it) with contractual assumptions of liability to the recipient, for
any liability that these contractual assumptions directly impose on
those licensors and authors.

All other non-permissive additional terms are considered "further
restrictions" within the meaning of section 10.  If the Program as you
received it, or any part of it, contains a notice stating that it is
governed by this License along with a term that is a further
restriction, you may remove that term.  If a license document contains
a further restriction but permits relicensing or conveying under this
License, you may add to a covered work material governed by the terms
of that license document, provided that the further restriction does
not survive such relicensing or conveying.

If you add terms to a covered work in accord with this section, you
must place, in the relevant source files, a statement of the
additional terms that apply to those files, or a notice indicating
where to find the applicable terms.

Additional terms, permissive or non-permissive, may be stated in the
form of a separately written license, or stated as exceptions;
the above requirements apply either way.

\vparasmall
8. Termination.

You may not propagate or modify a covered work except as expressly
provided under this License.  Any attempt otherwise to propagate or
modify it is void, and will automatically terminate your rights under
this License (including any patent licenses granted under the third
paragraph of section 11).

However, if you cease all violation of this License, then your
license from a particular copyright holder is reinstated (a)
provisionally, unless and until the copyright holder explicitly and
finally terminates your license, and (b) permanently, if the copyright
holder fails to notify you of the violation by some reasonable means
prior to 60 days after the cessation.

Moreover, your license from a particular copyright holder is
reinstated permanently if the copyright holder notifies you of the
violation by some reasonable means, this is the first time you have
received notice of violation of this License (for any work) from that
copyright holder, and you cure the violation prior to 30 days after
your receipt of the notice.

Termination of your rights under this section does not terminate the
licenses of parties who have received copies or rights from you under
this License.  If your rights have been terminated and not permanently
reinstated, you do not qualify to receive new licenses for the same
material under section 10.

\vparasmall
9. Acceptance Not Required for Having Copies.

You are not required to accept this License in order to receive or
run a copy of the Program.  Ancillary propagation of a covered work
occurring solely as a consequence of using peer-to-peer transmission
to receive a copy likewise does not require acceptance.  However,
nothing other than this License grants you permission to propagate or
modify any covered work.  These actions infringe copyright if you do
not accept this License.  Therefore, by modifying or propagating a
covered work, you indicate your acceptance of this License to do so.

\vparasmall
10. Automatic Licensing of Downstream Recipients.

Each time you convey a covered work, the recipient automatically
receives a license from the original licensors, to run, modify and
propagate that work, subject to this License.  You are not responsible
for enforcing compliance by third parties with this License.

An "entity transaction" is a transaction transferring control of an
organization, or substantially all assets of one, or subdividing an
organization, or merging organizations.  If propagation of a covered
work results from an entity transaction, each party to that
transaction who receives a copy of the work also receives whatever
licenses to the work the party's predecessor in interest had or could
give under the previous paragraph, plus a right to possession of the
Corresponding Source of the work from the predecessor in interest, if
the predecessor has it or can get it with reasonable efforts.

You may not impose any further restrictions on the exercise of the
rights granted or affirmed under this License.  For example, you may
not impose a license fee, royalty, or other charge for exercise of
rights granted under this License, and you may not initiate litigation
(including a cross-claim or counterclaim in a lawsuit) alleging that
any patent claim is infringed by making, using, selling, offering for
sale, or importing the Program or any portion of it.

\vparasmall
11. Patents.

A "contributor" is a copyright holder who authorizes use under this
License of the Program or a work on which the Program is based.  The
work thus licensed is called the contributor's "contributor version".

A contributor's "essential patent claims" are all patent claims
owned or controlled by the contributor, whether already acquired or
hereafter acquired, that would be infringed by some manner, permitted
by this License, of making, using, or selling its contributor version,
but do not include claims that would be infringed only as a
consequence of further modification of the contributor version.  For
purposes of this definition, "control" includes the right to grant
patent sublicenses in a manner consistent with the requirements of
this License.

Each contributor grants you a non-exclusive, worldwide, royalty-free
patent license under the contributor's essential patent claims, to
make, use, sell, offer for sale, import and otherwise run, modify and
propagate the contents of its contributor version.

In the following three paragraphs, a "patent license" is any express
agreement or commitment, however denominated, not to enforce a patent
(such as an express permission to practice a patent or covenant not to
sue for patent infringement).  To "grant" such a patent license to a
party means to make such an agreement or commitment not to enforce a
patent against the party.

If you convey a covered work, knowingly relying on a patent license,
and the Corresponding Source of the work is not available for anyone
to copy, free of charge and under the terms of this License, through a
publicly available network server or other readily accessible means,
then you must either (1) cause the Corresponding Source to be so
available, or (2) arrange to deprive yourself of the benefit of the
patent license for this particular work, or (3) arrange, in a manner
consistent with the requirements of this License, to extend the patent
license to downstream recipients.  "Knowingly relying" means you have
actual knowledge that, but for the patent license, your conveying the
covered work in a country, or your recipient's use of the covered work
in a country, would infringe one or more identifiable patents in that
country that you have reason to believe are valid.

If, pursuant to or in connection with a single transaction or
arrangement, you convey, or propagate by procuring conveyance of, a
covered work, and grant a patent license to some of the parties
receiving the covered work authorizing them to use, propagate, modify
or convey a specific copy of the covered work, then the patent license
you grant is automatically extended to all recipients of the covered
work and works based on it.

A patent license is "discriminatory" if it does not include within
the scope of its coverage, prohibits the exercise of, or is
conditioned on the non-exercise of one or more of the rights that are
specifically granted under this License.  You may not convey a covered
work if you are a party to an arrangement with a third party that is
in the business of distributing software, under which you make payment
to the third party based on the extent of your activity of conveying
the work, and under which the third party grants, to any of the
parties who would receive the covered work from you, a discriminatory
patent license (a) in connection with copies of the covered work
conveyed by you (or copies made from those copies), or (b) primarily
for and in connection with specific products or compilations that
contain the covered work, unless you entered into that arrangement,
or that patent license was granted, prior to 28 March 2007.

Nothing in this License shall be construed as excluding or limiting
any implied license or other defenses to infringement that may
otherwise be available to you under applicable patent law.

\vparasmall
12. No Surrender of Others' Freedom.

If conditions are imposed on you (whether by court order, agreement or
otherwise) that contradict the conditions of this License, they do not
excuse you from the conditions of this License.  If you cannot convey a
covered work so as to satisfy simultaneously your obligations under this
License and any other pertinent obligations, then as a consequence you may
not convey it at all.  For example, if you agree to terms that obligate you
to collect a royalty for further conveying from those to whom you convey
the Program, the only way you could satisfy both those terms and this
License would be to refrain entirely from conveying the Program.

\vparasmall
13. Use with the GNU Affero General Public License.

Notwithstanding any other provision of this License, you have
permission to link or combine any covered work with a work licensed
under version 3 of the GNU Affero General Public License into a single
combined work, and to convey the resulting work.  The terms of this
License will continue to apply to the part which is the covered work,
but the special requirements of the GNU Affero General Public License,
section 13, concerning interaction through a network will apply to the
combination as such.

\vparasmall
14. Revised Versions of this License.

The Free Software Foundation may publish revised and/or new versions of
the GNU General Public License from time to time.  Such new versions will
be similar in spirit to the present version, but may differ in detail to
address new problems or concerns.

Each version is given a distinguishing version number.  If the
Program specifies that a certain numbered version of the GNU General
Public License "or any later version" applies to it, you have the
option of following the terms and conditions either of that numbered
version or of any later version published by the Free Software
Foundation.  If the Program does not specify a version number of the
GNU General Public License, you may choose any version ever published
by the Free Software Foundation.

If the Program specifies that a proxy can decide which future
versions of the GNU General Public License can be used, that proxy's
public statement of acceptance of a version permanently authorizes you
to choose that version for the Program.

Later license versions may give you additional or different
permissions.  However, no additional obligations are imposed on any
author or copyright holder as a result of your choosing to follow a
later version.

\vparasmall
15. Disclaimer of Warranty.

THERE IS NO WARRANTY FOR THE PROGRAM, TO THE EXTENT PERMITTED BY
APPLICABLE LAW.  EXCEPT WHEN OTHERWISE STATED IN WRITING THE COPYRIGHT
HOLDERS AND/OR OTHER PARTIES PROVIDE THE PROGRAM "AS IS" WITHOUT WARRANTY
OF ANY KIND, EITHER EXPRESSED OR IMPLIED, INCLUDING, BUT NOT LIMITED TO,
THE IMPLIED WARRANTIES OF MERCHANTABILITY AND FITNESS FOR A PARTICULAR
PURPOSE.  THE ENTIRE RISK AS TO THE QUALITY AND PERFORMANCE OF THE PROGRAM
IS WITH YOU.  SHOULD THE PROGRAM PROVE DEFECTIVE, YOU ASSUME THE COST OF
ALL NECESSARY SERVICING, REPAIR OR CORRECTION.

\vparasmall
16. Limitation of Liability.

IN NO EVENT UNLESS REQUIRED BY APPLICABLE LAW OR AGREED TO IN WRITING
WILL ANY COPYRIGHT HOLDER, OR ANY OTHER PARTY WHO MODIFIES AND/OR CONVEYS
THE PROGRAM AS PERMITTED ABOVE, BE LIABLE TO YOU FOR DAMAGES, INCLUDING ANY
GENERAL, SPECIAL, INCIDENTAL OR CONSEQUENTIAL DAMAGES ARISING OUT OF THE
USE OR INABILITY TO USE THE PROGRAM (INCLUDING BUT NOT LIMITED TO LOSS OF
DATA OR DATA BEING RENDERED INACCURATE OR LOSSES SUSTAINED BY YOU OR THIRD
PARTIES OR A FAILURE OF THE PROGRAM TO OPERATE WITH ANY OTHER PROGRAMS),
EVEN IF SUCH HOLDER OR OTHER PARTY HAS BEEN ADVISED OF THE POSSIBILITY OF
SUCH DAMAGES.

\vparasmall
17. Interpretation of Sections 15 and 16.

If the disclaimer of warranty and limitation of liability provided
above cannot be given local legal effect according to their terms,
reviewing courts shall apply local law that most closely approximates
an absolute waiver of all civil liability in connection with the
Program, unless a warranty or assumption of liability accompanies a
copy of the Program in return for a fee.

\vparasmall
END OF TERMS AND CONDITIONS \\

\vparasmall
How to Apply These Terms to Your New Programs

\vparasmall
If you develop a new program, and you want it to be of the greatest
possible use to the public, the best way to achieve this is to make it
free software which everyone can redistribute and change under these terms.

To do so, attach the following notices to the program.  It is safest
to attach them to the start of each source file to most effectively
state the exclusion of warranty; and each file should have at least
the "copyright" line and a pointer to where the full notice is found.

\vparasmall
$<$one line to give the program's name and a brief idea of what it does.$>$
Copyright (C) $<$year$>$  $<$name of author$>$

This program is free software: you can redistribute it and/or modify
it under the terms of the GNU General Public License as published by
the Free Software Foundation, either version 3 of the License, or
(at your option) any later version.

This program is distributed in the hope that it will be useful,
but WITHOUT ANY WARRANTY; without even the implied warranty of
MERCHANTABILITY or FITNESS FOR A PARTICULAR PURPOSE.  See the
GNU General Public License for more details.

You should have received a copy of the GNU General Public License
along with this program.  If not, see 
$<$\href{http://www.gnu.org/licenses/}{http://www.gnu.org/licenses/}$>$ 

\vparasmall
Also add information on how to contact you by electronic and paper mail.

If the program does terminal interaction, make it output a short
notice like this when it starts in an interactive mode:

$<$program$>$  Copyright (C) $<$year$>$  $<$name of author$>$
This program comes with ABSOLUTELY NO WARRANTY; for details type `show w'.
This is free software, and you are welcome to redistribute it
under certain conditions; type `show c' for details.

The hypothetical commands `show w' and `show c' should show the appropriate
parts of the General Public License.  Of course, your program's commands
might be different; for a GUI interface, you would use an "about box".

You should also get your employer (if you work as a programmer) or school,
if any, to sign a "copyright disclaimer" for the program, if necessary.
For more information on this, and how to apply and follow the GNU GPL, see
$<$\href{http://www.gnu.org/licenses/}{http://www.gnu.org/licenses/}$>$ 

The GNU General Public License does not permit incorporating your program
into proprietary programs.  If your program is a subroutine library, you
may consider it more useful to permit linking proprietary applications with
the library.  If this is what you want to do, use the GNU Lesser General
Public License instead of this License.  But first, please read

$<$\href{http://www.gnu.org/philosophy/why-not-lgpl.html}{http://www.gnu.org/philosophy/why-not-lgpl.html}$>$ 


\newpage

\subsection{GNU Free Documentation License, Version 1.3}
\label{GNUFDL}
\begin{center}
	GNU Free Documentation License
	
	1.3, 3 November 2008
\end{center}

\noindent Copyright (C) 2000, 2001, 2002, 2007, 2008 Free Software Foundation, Inc.  $<$\href{http://fsf.org/}{http://fsf.org/}$>$ \\

\noindent Everyone is permitted to copy and distribute verbatim copies
of this license document, but changing it is not allowed. \\



0. PREAMBLE

The purpose of this License is to make a manual, textbook, or other functional and useful document "free" in the sense of freedom: to assure everyone the effective freedom to copy and redistribute it, with or without modifying it, either commercially or noncommercially. Secondarily, this License preserves for the author and publisher a way to get credit for their work, while not being considered responsible for modifications made by others.

This License is a kind of "copyleft", which means that derivative works of the document must themselves be free in the same sense. It complements the GNU General Public License, which is a copyleft license designed for free software.

We have designed this License in order to use it for manuals for free software, because free software needs free documentation: a free program should come with manuals providing the same freedoms that the software does. But this License is not limited to software manuals; it can be used for any textual work, regardless of subject matter or whether it is published as a printed book. We recommend this License principally for works whose purpose is instruction or reference.

\vparasmall
1. APPLICABILITY AND DEFINITIONS


This License applies to any manual or other work, in any medium, that contains a notice placed by the copyright holder saying it can be distributed under the terms of this License. Such a notice grants a world-wide, royalty-free license, unlimited in duration, to use that work under the conditions stated herein. The "Document", below, refers to any such manual or work. Any member of the public is a licensee, and is addressed as "you". You accept the license if you copy, modify or distribute the work in a way requiring permission under copyright law.

A "Modified Version" of the Document means any work containing the Document or a portion of it, either copied verbatim, or with modifications and/or translated into another language.

A "Secondary Section" is a named appendix or a front-matter section of the Document that deals exclusively with the relationship of the publishers or authors of the Document to the Document's overall subject (or to related matters) and contains nothing that could fall directly within that overall subject. (Thus, if the Document is in part a textbook of mathematics, a Secondary Section may not explain any mathematics.) The relationship could be a matter of historical connection with the subject or with related matters, or of legal, commercial, philosophical, ethical or political position regarding them.

The "Invariant Sections" are certain Secondary Sections whose titles are designated, as being those of Invariant Sections, in the notice that says that the Document is released under this License. If a section does not fit the above definition of Secondary then it is not allowed to be designated as Invariant. The Document may contain zero Invariant Sections. If the Document does not identify any Invariant Sections then there are none.

The "Cover Texts" are certain short passages of text that are listed, as Front-Cover Texts or Back-Cover Texts, in the notice that says that the Document is released under this License. A Front-Cover Text may be at most 5 words, and a Back-Cover Text may be at most 25 words.

A "Transparent" copy of the Document means a machine-readable copy, represented in a format whose specification is available to the general public, that is suitable for revising the document straightforwardly with generic text editors or (for images composed of pixels) generic paint programs or (for drawings) some widely available drawing editor, and that is suitable for input to text formatters or for automatic translation to a variety of formats suitable for input to text formatters. A copy made in an otherwise Transparent file format whose markup, or absence of markup, has been arranged to thwart or discourage subsequent modification by readers is not Transparent. An image format is not Transparent if used for any substantial amount of text. A copy that is not "Transparent" is called "Opaque".

Examples of suitable formats for Transparent copies include plain ASCII without markup, Texinfo input format, LaTeX input format, SGML or XML using a publicly available DTD, and standard-conforming simple HTML, PostScript or PDF designed for human modification. Examples of transparent image formats include PNG, XCF and JPG. Opaque formats include proprietary formats that can be read and edited only by proprietary word processors, SGML or XML for which the DTD and/or processing tools are not generally available, and the machine-generated HTML, PostScript or PDF produced by some word processors for output purposes only.

The "Title Page" means, for a printed book, the title page itself, plus such following pages as are needed to hold, legibly, the material this License requires to appear in the title page. For works in formats which do not have any title page as such, "Title Page" means the text near the most prominent appearance of the work's title, preceding the beginning of the body of the text.

The "publisher" means any person or entity that distributes copies of the Document to the public.

A section "Entitled XYZ" means a named subunit of the Document whose title either is precisely XYZ or contains XYZ in parentheses following text that translates XYZ in another language. (Here XYZ stands for a specific section name mentioned below, such as "Acknowledgements", "Dedications", "Endorsements", or "History".) To "Preserve the Title" of such a section when you modify the Document means that it remains a section "Entitled XYZ" according to this definition.

The Document may include Warranty Disclaimers next to the notice which states that this License applies to the Document. These Warranty Disclaimers are considered to be included by reference in this License, but only as regards disclaiming warranties: any other implication that these Warranty Disclaimers may have is void and has no effect on the meaning of this License.

\vparasmall
2. VERBATIM COPYING

You may copy and distribute the Document in any medium, either commercially or noncommercially, provided that this License, the copyright notices, and the license notice saying this License applies to the Document are reproduced in all copies, and that you add no other conditions whatsoever to those of this License. You may not use technical measures to obstruct or control the reading or further copying of the copies you make or distribute. However, you may accept compensation in exchange for copies. If you distribute a large enough number of copies you must also follow the conditions in section 3.

You may also lend copies, under the same conditions stated above, and you may publicly display copies.

\vparasmall
3. COPYING IN QUANTITY

If you publish printed copies (or copies in media that commonly have printed covers) of the Document, numbering more than 100, and the Document's license notice requires Cover Texts, you must enclose the copies in covers that carry, clearly and legibly, all these Cover Texts: Front-Cover Texts on the front cover, and Back-Cover Texts on the back cover. Both covers must also clearly and legibly identify you as the publisher of these copies. The front cover must present the full title with all words of the title equally prominent and visible. You may add other material on the covers in addition. Copying with changes limited to the covers, as long as they preserve the title of the Document and satisfy these conditions, can be treated as verbatim copying in other respects.

If the required texts for either cover are too voluminous to fit legibly, you should put the first ones listed (as many as fit reasonably) on the actual cover, and continue the rest onto adjacent pages.

If you publish or distribute Opaque copies of the Document numbering more than 100, you must either include a machine-readable Transparent copy along with each Opaque copy, or state in or with each Opaque copy a computer-network location from which the general network-using public has access to download using public-standard network protocols a complete Transparent copy of the Document, free of added material. If you use the latter option, you must take reasonably prudent steps, when you begin distribution of Opaque copies in quantity, to ensure that this Transparent copy will remain thus accessible at the stated location until at least one year after the last time you distribute an Opaque copy (directly or through your agents or retailers) of that edition to the public.

It is requested, but not required, that you contact the authors of the Document well before redistributing any large number of copies, to give them a chance to provide you with an updated version of the Document.

\vparasmall
4. MODIFICATIONS

You may copy and distribute a Modified Version of the Document under the conditions of sections 2 and 3 above, provided that you release the Modified Version under precisely this License, with the Modified Version filling the role of the Document, thus licensing distribution and modification of the Modified Version to whoever possesses a copy of it. In addition, you must do these things in the Modified Version:

A. Use in the Title Page (and on the covers, if any) a title distinct from that of the Document, and from those of previous versions (which should, if there were any, be listed in the History section of the Document). You may use the same title as a previous version if the original publisher of that version gives permission. 

B. List on the Title Page, as authors, one or more persons or entities responsible for authorship of the modifications in the Modified Version, together with at least five of the principal authors of the Document (all of its principal authors, if it has fewer than five), unless they release you from this requirement. 

C. State on the Title page the name of the publisher of the Modified Version, as the publisher. 

D. Preserve all the copyright notices of the Document. 

E. Add an appropriate copyright notice for your modifications adjacent to the other copyright notices. 

F. Include, immediately after the copyright notices, a license notice giving the public permission to use the Modified Version under the terms of this License, in the form shown in the Addendum below. 

G. Preserve in that license notice the full lists of Invariant Sections and required Cover Texts given in the Document's license notice. 

H. Include an unaltered copy of this License. 

I. Preserve the section Entitled "History", Preserve its Title, and add to it an item stating at least the title, year, new authors, and publisher of the Modified Version as given on the Title Page. If there is no section Entitled "History" in the Document, create one stating the title, year, authors, and publisher of the Document as given on its Title Page, then add an item describing the Modified Version as stated in the previous sentence. 

J. Preserve the network location, if any, given in the Document for public access to a Transparent copy of the Document, and likewise the network locations given in the Document for previous versions it was based on. These may be placed in the "History" section. You may omit a network location for a work that was published at least four years before the Document itself, or if the original publisher of the version it refers to gives permission. 

K. For any section Entitled "Acknowledgements" or "Dedications", Preserve the Title of the section, and preserve in the section all the substance and tone of each of the contributor acknowledgements and/or dedications given therein. 

L. Preserve all the Invariant Sections of the Document, unaltered in their text and in their titles. Section numbers or the equivalent are not considered part of the section titles. 

M. Delete any section Entitled "Endorsements". Such a section may not be included in the Modified Version. 

N. Do not retitle any existing section to be Entitled "Endorsements" or to conflict in title with any Invariant Section. 

O. Preserve any Warranty Disclaimers. 

If the Modified Version includes new front-matter sections or appendices that qualify as Secondary Sections and contain no material copied from the Document, you may at your option designate some or all of these sections as invariant. To do this, add their titles to the list of Invariant Sections in the Modified Version's license notice. These titles must be distinct from any other section titles.

You may add a section Entitled "Endorsements", provided it contains nothing but endorsements of your Modified Version by various parties-for example, statements of peer review or that the text has been approved by an organization as the authoritative definition of a standard.

You may add a passage of up to five words as a Front-Cover Text, and a passage of up to 25 words as a Back-Cover Text, to the end of the list of Cover Texts in the Modified Version. Only one passage of Front-Cover Text and one of Back-Cover Text may be added by (or through arrangements made by) any one entity. If the Document already includes a cover text for the same cover, previously added by you or by arrangement made by the same entity you are acting on behalf of, you may not add another; but you may replace the old one, on explicit permission from the previous publisher that added the old one.

The author(s) and publisher(s) of the Document do not by this License give permission to use their names for publicity for or to assert or imply endorsement of any Modified Version.

\vparasmall
5. COMBINING DOCUMENTS

You may combine the Document with other documents released under this License, under the terms defined in section 4 above for modified versions, provided that you include in the combination all of the Invariant Sections of all of the original documents, unmodified, and list them all as Invariant Sections of your combined work in its license notice, and that you preserve all their Warranty Disclaimers.

The combined work need only contain one copy of this License, and multiple identical Invariant Sections may be replaced with a single copy. If there are multiple Invariant Sections with the same name but different contents, make the title of each such section unique by adding at the end of it, in parentheses, the name of the original author or publisher of that section if known, or else a unique number. Make the same adjustment to the section titles in the list of Invariant Sections in the license notice of the combined work.

In the combination, you must combine any sections Entitled "History" in the various original documents, forming one section Entitled "History"; likewise combine any sections Entitled "Acknowledgements", and any sections Entitled "Dedications". You must delete all sections Entitled "Endorsements".

\vparasmall
6. COLLECTIONS OF DOCUMENTS

You may make a collection consisting of the Document and other documents released under this License, and replace the individual copies of this License in the various documents with a single copy that is included in the collection, provided that you follow the rules of this License for verbatim copying of each of the documents in all other respects.

You may extract a single document from such a collection, and distribute it individually under this License, provided you insert a copy of this License into the extracted document, and follow this License in all other respects regarding verbatim copying of that document.

\vparasmall
7. AGGREGATION WITH INDEPENDENT WORKS

A compilation of the Document or its derivatives with other separate and independent documents or works, in or on a volume of a storage or distribution medium, is called an "aggregate" if the copyright resulting from the compilation is not used to limit the legal rights of the compilation's users beyond what the individual works permit. When the Document is included in an aggregate, this License does not apply to the other works in the aggregate which are not themselves derivative works of the Document.

If the Cover Text requirement of section 3 is applicable to these copies of the Document, then if the Document is less than one half of the entire aggregate, the Document's Cover Texts may be placed on covers that bracket the Document within the aggregate, or the electronic equivalent of covers if the Document is in electronic form. Otherwise they must appear on printed covers that bracket the whole aggregate.

\vparasmall
8. TRANSLATION

Translation is considered a kind of modification, so you may distribute translations of the Document under the terms of section 4. Replacing Invariant Sections with translations requires special permission from their copyright holders, but you may include translations of some or all Invariant Sections in addition to the original versions of these Invariant Sections. You may include a translation of this License, and all the license notices in the Document, and any Warranty Disclaimers, provided that you also include the original English version of this License and the original versions of those notices and disclaimers. In case of a disagreement between the translation and the original version of this License or a notice or disclaimer, the original version will prevail.

If a section in the Document is Entitled "Acknowledgements", "Dedications", or "History", the requirement (section 4) to Preserve its Title (section 1) will typically require changing the actual title.

\vparasmall
9. TERMINATION

You may not copy, modify, sublicense, or distribute the Document except as expressly provided under this License. Any attempt otherwise to copy, modify, sublicense, or distribute it is void, and will automatically terminate your rights under this License.

However, if you cease all violation of this License, then your license from a particular copyright holder is reinstated (a) provisionally, unless and until the copyright holder explicitly and finally terminates your license, and (b) permanently, if the copyright holder fails to notify you of the violation by some reasonable means prior to 60 days after the cessation.

Moreover, your license from a particular copyright holder is reinstated permanently if the copyright holder notifies you of the violation by some reasonable means, this is the first time you have received notice of violation of this License (for any work) from that copyright holder, and you cure the violation prior to 30 days after your receipt of the notice.

Termination of your rights under this section does not terminate the licenses of parties who have received copies or rights from you under this License. If your rights have been terminated and not permanently reinstated, receipt of a copy of some or all of the same material does not give you any rights to use it.

\vparasmall
10. FUTURE REVISIONS OF THIS LICENSE

The Free Software Foundation may publish new, revised versions of the GNU Free Documentation License from time to time. Such new versions will be similar in spirit to the present version, but may differ in detail to address new problems or concerns. See http://www.gnu.org/copyleft/.

Each version of the License is given a distinguishing version number. If the Document specifies that a particular numbered version of this License "or any later version" applies to it, you have the option of following the terms and conditions either of that specified version or of any later version that has been published (not as a draft) by the Free Software Foundation. If the Document does not specify a version number of this License, you may choose any version ever published (not as a draft) by the Free Software Foundation. If the Document specifies that a proxy can decide which future versions of this License can be used, that proxy's public statement of acceptance of a version permanently authorizes you to choose that version for the Document.

\vparasmall
11. RELICENSING

"Massive Multiauthor Collaboration Site" (or "MMC Site") means any World Wide Web server that publishes copyrightable works and also provides prominent facilities for anybody to edit those works. A public wiki that anybody can edit is an example of such a server. A "Massive Multiauthor Collaboration" (or "MMC") contained in the site means any set of copyrightable works thus published on the MMC site.

"CC-BY-SA" means the Creative Commons Attribution-Share Alike 3.0 license published by Creative Commons Corporation, a not-for-profit corporation with a principal place of business in San Francisco, California, as well as future copyleft versions of that license published by that same organization.

"Incorporate" means to publish or republish a Document, in whole or in part, as part of another Document.

An MMC is "eligible for relicensing" if it is licensed under this License, and if all works that were first published under this License somewhere other than this MMC, and subsequently incorporated in whole or in part into the MMC, (1) had no cover texts or invariant sections, and (2) were thus incorporated prior to November 1, 2008.

The operator of an MMC Site may republish an MMC contained in the site under CC-BY-SA on the same site at any time before August 1, 2009, provided the MMC is eligible for relicensing.

\vparasmall
ADDENDUM: How to use this License for your documents

To use this License in a document you have written, include a copy of the License in the document and put the following copyright and license notices just after the title page:

Copyright (C)  YEAR  YOUR NAME.
Permission is granted to copy, distribute and/or modify this document
under the terms of the GNU Free Documentation License, Version 1.3
or any later version published by the Free Software Foundation;
with no Invariant Sections, no Front-Cover Texts, and no Back-Cover Texts.
A copy of the license is included in the section entitled "GNU
Free Documentation License".

If you have Invariant Sections, Front-Cover Texts and Back-Cover Texts, replace the "with ? Texts." line with this:

with the Invariant Sections being LIST THEIR TITLES, with the
Front-Cover Texts being LIST, and with the Back-Cover Texts being LIST.

If you have Invariant Sections without Cover Texts, or some other combination of the three, merge those two alternatives to suit the situation.

If your document contains nontrivial examples of program code, we recommend releasing these examples in parallel under your choice of free software license, such as the GNU General Public License, to permit their use in free software. 






\newpage
\section{Other Licenses}
\subsection{Mozilla Public License, Version 2.0}
\label{MPL2}
\begin{center}
	Mozilla Public License
	Version 2.0
\end{center}


1. Definitions

1.1. ''Contributor''

means each individual or legal entity that creates, contributes to the creation of, or owns Covered Software.

1.2. ''Contributor Version''

means the combination of the Contributions of others (if any) used by a Contributor and that particular Contributor’s Contribution.

1.3. ''Contribution''

means Covered Software of a particular Contributor.

1.4. ''Covered Software''

means Source Code Form to which the initial Contributor has attached the notice in Exhibit A, the Executable Form of such Source Code Form, and Modifications of such Source Code Form, in each case including portions thereof.

1.5. ''Incompatible With Secondary Licenses''

means

a.that the initial Contributor has attached the notice described in Exhibit B to the Covered Software; or

b.that the Covered Software was made available under the terms of version 1.1 or earlier of the License, but not also under the terms of a Secondary License.

1.6. ''Executable Form''

means any form of the work other than Source Code Form.

1.7. ''Larger Work''

means a work that combines Covered Software with other material, in a separate file or files, that is not Covered Software.

1.8. ''License''

means this document.

1.9. ''Licensable''

means having the right to grant, to the maximum extent possible, whether at the time of the initial grant or subsequently, any and all of the rights conveyed by this License.

1.10. ''Modifications''

means any of the following:

a.any file in Source Code Form that results from an addition to, deletion from, or modification of the contents of Covered Software; or

b.any new file in Source Code Form that contains any Covered Software.

1.11. ''Patent Claims'' of a Contributor

means any patent claim(s), including without limitation, method, process, and apparatus claims, in any patent Licensable by such Contributor that would be infringed, but for the grant of the License, by the making, using, selling, offering for sale, having made, import, or transfer of either its Contributions or its Contributor Version.

1.12. ''Secondary License''

means either the GNU General Public License, Version 2.0, the GNU Lesser General Public License, Version 2.1, the GNU Affero General Public License, Version 3.0, or any later versions of those licenses.

1.13. ''Source Code Form''

means the form of the work preferred for making modifications.

1.14. ''You'' (or ''Your'')

means an individual or a legal entity exercising rights under this License. For legal entities, ''You'' includes any entity that controls, is controlled by, or is under common control with You. For purposes of this definition, ''control'' means (a) the power, direct or indirect, to cause the direction or management of such entity, whether by contract or otherwise, or (b) ownership of more than fifty percent (50\%) of the outstanding shares or beneficial ownership of such entity.

\vparasmall
2. License Grants and Conditions

2.1. Grants

Each Contributor hereby grants You a world-wide, royalty-free, non-exclusive license:

a.under intellectual property rights (other than patent or trademark) Licensable by such Contributor to use, reproduce, make available, modify, display, perform, distribute, and otherwise exploit its Contributions, either on an unmodified basis, with Modifications, or as part of a Larger Work; and

b.under Patent Claims of such Contributor to make, use, sell, offer for sale, have made, import, and otherwise transfer either its Contributions or its Contributor Version.

2.2. Effective Date

The licenses granted in Section 2.1 with respect to any Contribution become effective for each Contribution on the date the Contributor first distributes such Contribution.

2.3. Limitations on Grant Scope

The licenses granted in this Section 2 are the only rights granted under this License. No additional rights or licenses will be implied from the distribution or licensing of Covered Software under this License. Notwithstanding Section 2.1(b) above, no patent license is granted by a Contributor:

a.for any code that a Contributor has removed from Covered Software; or

b.for infringements caused by: (i) Your and any other third party’s modifications of Covered Software, or (ii) the combination of its Contributions with other software (except as part of its Contributor Version); or

c.under Patent Claims infringed by Covered Software in the absence of its Contributions.

This License does not grant any rights in the trademarks, service marks, or logos of any Contributor (except as may be necessary to comply with the notice requirements in Section 3.4).

2.4. Subsequent Licenses

No Contributor makes additional grants as a result of Your choice to distribute the Covered Software under a subsequent version of this License (see Section 10.2) or under the terms of a Secondary License (if permitted under the terms of Section 3.3).

2.5. Representation

Each Contributor represents that the Contributor believes its Contributions are its original creation(s) or it has sufficient rights to grant the rights to its Contributions conveyed by this License.

2.6. Fair Use

This License is not intended to limit any rights You have under applicable copyright doctrines of fair use, fair dealing, or other equivalents.

2.7. Conditions

Sections 3.1, 3.2, 3.3, and 3.4 are conditions of the licenses granted in Section 2.1.

\vparasmall
3. Responsibilities

3.1. Distribution of Source Form

All distribution of Covered Software in Source Code Form, including any Modifications that You create or to which You contribute, must be under the terms of this License. You must inform recipients that the Source Code Form of the Covered Software is governed by the terms of this License, and how they can obtain a copy of this License. You may not attempt to alter or restrict the recipients’ rights in the Source Code Form.

3.2. Distribution of Executable Form

If You distribute Covered Software in Executable Form then:

a.such Covered Software must also be made available in Source Code Form, as described in Section 3.1, and You must inform recipients of the Executable Form how they can obtain a copy of such Source Code Form by reasonable means in a timely manner, at a charge no more than the cost of distribution to the recipient; and

b.You may distribute such Executable Form under the terms of this License, or sublicense it under different terms, provided that the license for the Executable Form does not attempt to limit or alter the recipients’ rights in the Source Code Form under this License.

3.3. Distribution of a Larger Work

You may create and distribute a Larger Work under terms of Your choice, provided that You also comply with the requirements of this License for the Covered Software. If the Larger Work is a combination of Covered Software with a work governed by one or more Secondary Licenses, and the Covered Software is not Incompatible With Secondary Licenses, this License permits You to additionally distribute such Covered Software under the terms of such Secondary License(s), so that the recipient of the Larger Work may, at their option, further distribute the Covered Software under the terms of either this License or such Secondary License(s).

3.4. Notices

You may not remove or alter the substance of any license notices (including copyright notices, patent notices, disclaimers of warranty, or limitations of liability) contained within the Source Code Form of the Covered Software, except that You may alter any license notices to the extent required to remedy known factual inaccuracies.

3.5. Application of Additional Terms

You may choose to offer, and to charge a fee for, warranty, support, indemnity or liability obligations to one or more recipients of Covered Software. However, You may do so only on Your own behalf, and not on behalf of any Contributor. You must make it absolutely clear that any such warranty, support, indemnity, or liability obligation is offered by You alone, and You hereby agree to indemnify every Contributor for any liability incurred by such Contributor as a result of warranty, support, indemnity or liability terms You offer. You may include additional disclaimers of warranty and limitations of liability specific to any jurisdiction.

\vparasmall
4. Inability to Comply Due to Statute or Regulation

If it is impossible for You to comply with any of the terms of this License with respect to some or all of the Covered Software due to statute, judicial order, or regulation then You must: (a) comply with the terms of this License to the maximum extent possible; and (b) describe the limitations and the code they affect. Such description must be placed in a text file included with all distributions of the Covered Software under this License. Except to the extent prohibited by statute or regulation, such description must be sufficiently detailed for a recipient of ordinary skill to be able to understand it.

\vparasmall
5. Termination

5.1. The rights granted under this License will terminate automatically if You fail to comply with any of its terms. However, if You become compliant, then the rights granted under this License from a particular Contributor are reinstated (a) provisionally, unless and until such Contributor explicitly and finally terminates Your grants, and (b) on an ongoing basis, if such Contributor fails to notify You of the non-compliance by some reasonable means prior to 60 days after You have come back into compliance. Moreover, Your grants from a particular Contributor are reinstated on an ongoing basis if such Contributor notifies You of the non-compliance by some reasonable means, this is the first time You have received notice of non-compliance with this License from such Contributor, and You become compliant prior to 30 days after Your receipt of the notice.

5.2. If You initiate litigation against any entity by asserting a patent infringement claim (excluding declaratory judgment actions, counter-claims, and cross-claims) alleging that a Contributor Version directly or indirectly infringes any patent, then the rights granted to You by any and all Contributors for the Covered Software under Section 2.1 of this License shall terminate.

5.3. In the event of termination under Sections 5.1 or 5.2 above, all end user license agreements (excluding distributors and resellers) which have been validly granted by You or Your distributors under this License prior to termination shall survive termination.

\vparasmall
6. Disclaimer of Warranty

Covered Software is provided under this License on an ''as is'' basis, without warranty of any kind, either expressed, implied, or statutory, including, without limitation, warranties that the Covered Software is free of defects, merchantable, fit for a particular purpose or non-infringing. The entire risk as to the quality and performance of the Covered Software is with You. Should any Covered Software prove defective in any respect, You (not any Contributor) assume the cost of any necessary servicing, repair, or correction. This disclaimer of warranty constitutes an essential part of this License. No use of any Covered Software is authorized under this License except under this disclaimer.

\vparasmall
7. Limitation of Liability

Under no circumstances and under no legal theory, whether tort (including negligence), contract, or otherwise, shall any Contributor, or anyone who distributes Covered Software as permitted above, be liable to You for any direct, indirect, special, incidental, or consequential damages of any character including, without limitation, damages for lost profits, loss of goodwill, work stoppage, computer failure or malfunction, or any and all other commercial damages or losses, even if such party shall have been informed of the possibility of such damages. This limitation of liability shall not apply to liability for death or personal injury resulting from such party’s negligence to the extent applicable law prohibits such limitation. Some jurisdictions do not allow the exclusion or limitation of incidental or consequential damages, so this exclusion and limitation may not apply to You.

\vparasmall
8. Litigation

Any litigation relating to this License may be brought only in the courts of a jurisdiction where the defendant maintains its principal place of business and such litigation shall be governed by laws of that jurisdiction, without reference to its conflict-of-law provisions. Nothing in this Section shall prevent a party’s ability to bring cross-claims or counter-claims.

\vparasmall
9. Miscellaneous

This License represents the complete agreement concerning the subject matter hereof. If any provision of this License is held to be unenforceable, such provision shall be reformed only to the extent necessary to make it enforceable. Any law or regulation which provides that the language of a contract shall be construed against the drafter shall not be used to construe this License against a Contributor.

\vparasmall
10. Versions of the License

10.1. New Versions

Mozilla Foundation is the license steward. Except as provided in Section 10.3, no one other than the license steward has the right to modify or publish new versions of this License. Each version will be given a distinguishing version number.

10.2. Effect of New Versions

You may distribute the Covered Software under the terms of the version of the License under which You originally received the Covered Software, or under the terms of any subsequent version published by the license steward.

10.3. Modified Versions

If you create software not governed by this License, and you want to create a new license for such software, you may create and use a modified version of this License if you rename the license and remove any references to the name of the license steward (except to note that such modified license differs from this License).

10.4. Distributing Source Code Form that is Incompatible With Secondary Licenses

If You choose to distribute Source Code Form that is Incompatible With Secondary Licenses under the terms of this version of the License, the notice described in Exhibit B of this License must be attached.

\vparasmall
Exhibit A - Source Code Form License Notice

This Source Code Form is subject to the terms of the Mozilla Public License, v. 2.0. If a copy of the MPL was not distributed with this file, You can obtain one at http://mozilla.org/MPL/2.0/.

If it is not possible or desirable to put the notice in a particular file, then You may include the notice in a location (such as a LICENSE file in a relevant directory) where a recipient would be likely to look for such a notice.

You may add additional accurate notices of copyright ownership.

\vparasmall
Exhibit B - ''Incompatible With Secondary Licenses'' Notice

This Source Code Form is ''Incompatible With Secondary Licenses'', as defined by the Mozilla Public License, v. 2.0.







%
%
%\newpage
%\subsection{The Code Project Open License (CPOL), Version 1.02}
%\label{CPOL}
%\begin{center}
%The Code Project Open License (CPOL) 
%Version 1.02
%\end{center}
%
%
%
%Preamble
%
%This License governs Your use of the Work. This License is intended to allow developers to use the Source Code and Executable Files provided as part of the Work in any application in any form. 
%
%The main points subject to the terms of the License are:
%
%Source Code and Executable Files can be used in commercial applications; 
%
%Source Code and Executable Files can be redistributed; and 
%
%Source Code can be modified to create derivative works. 
%
%No claim of suitability, guarantee, or any warranty whatsoever is provided. The software is provided "as-is". 
%
%The Article accompanying the Work may not be distributed or republished without the Author's consent 
%
%This License is entered between You, the individual or other entity reading or otherwise making use of the Work licensed pursuant to this License and the individual or other entity which offers the Work under the terms of this License ("Author").
%
%\vparasmall
%License
%
%THE WORK (AS DEFINED BELOW) IS PROVIDED UNDER THE TERMS OF THIS CODE PROJECT OPEN LICENSE ("LICENSE"). THE WORK IS PROTECTED BY COPYRIGHT AND/OR OTHER APPLICABLE LAW. ANY USE OF THE WORK OTHER THAN AS AUTHORIZED UNDER THIS LICENSE OR COPYRIGHT LAW IS PROHIBITED.
%
%BY EXERCISING ANY RIGHTS TO THE WORK PROVIDED HEREIN, YOU ACCEPT AND AGREE TO BE BOUND BY THE TERMS OF THIS LICENSE. THE AUTHOR GRANTS YOU THE RIGHTS CONTAINED HEREIN IN CONSIDERATION OF YOUR ACCEPTANCE OF SUCH TERMS AND CONDITIONS. IF YOU DO NOT AGREE TO ACCEPT AND BE BOUND BY THE TERMS OF THIS LICENSE, YOU CANNOT MAKE ANY USE OF THE WORK.
%
%\vparasmall
%Definitions. 
%
%"Articles" means, collectively, all articles written by Author which describes how the Source Code and Executable Files for the Work may be used by a user. 
%
%"Author" means the individual or entity that offers the Work under the terms of this License. 
%
%"Derivative Work" means a work based upon the Work or upon the Work and other pre-existing works. 
%
%"Executable Files" refer to the executables, binary files, configuration and any required data files included in the Work. 
%
%"Publisher" means the provider of the website, magazine, CD-ROM, DVD or other medium from or by which the Work is obtained by You. 
%
%"Source Code" refers to the collection of source code and configuration files used to create the Executable Files. 
%
%"Standard Version" refers to such a Work if it has not been modified, or has been modified in accordance with the consent of the Author, such consent being in the full discretion of the Author. 
%
%"Work" refers to the collection of files distributed by the Publisher, including the Source Code, Executable Files, binaries, data files, documentation, whitepapers and the Articles. 
%
%"You" is you, an individual or entity wishing to use the Work and exercise your rights under this License. 
%
%Fair Use/Fair Use Rights. Nothing in this License is intended to reduce, limit, or restrict any rights arising from fair use, fair dealing, first sale or other limitations on the exclusive rights of the copyright owner under copyright law or other applicable laws. 
%
%\vparasmall
%License Grant. 
%
%Subject to the terms and conditions of this License, the Author hereby grants You a worldwide, royalty-free, non-exclusive, perpetual (for the duration of the applicable copyright) license to exercise the rights in the Work as stated below: 
%
%You may use the standard version of the Source Code or Executable Files in Your own applications. 
%
%You may apply bug fixes, portability fixes and other modifications obtained from the Public Domain or from the Author. A Work modified in such a way shall still be considered the standard version and will be subject to this License. 
%
%You may otherwise modify Your copy of this Work (excluding the Articles) in any way to create a Derivative Work, provided that You insert a prominent notice in each changed file stating how, when and where You changed that file. 
%
%You may distribute the standard version of the Executable Files and Source Code or Derivative Work in aggregate with other (possibly commercial) programs as part of a larger (possibly commercial) software distribution. 
%The Articles discussing the Work published in any form by the author may not be distributed or republished without the Author's consent. The author retains copyright to any such Articles. You may use the Executable Files and Source Code pursuant to this License but you may not repost or republish or otherwise distribute or make available the Articles, without the prior written consent of the Author. 
%
%Any subroutines or modules supplied by You and linked into the Source Code or Executable Files of this Work shall not be considered part of this Work and will not be subject to the terms of this License. 
%
%\vparasmall
%Patent License. 
%
%Subject to the terms and conditions of this License, each Author hereby grants to You a perpetual, worldwide, non-exclusive, no-charge, royalty-free, irrevocable (except as stated in this section) patent license to make, have made, use, import, and otherwise transfer the Work. 
%
%\vparasmall
%Restrictions. 
%
%The license granted in Section 3 above is expressly made subject to and limited by the following restrictions: 
%
%You agree not to remove any of the original copyright, patent, trademark, and attribution notices and associated disclaimers that may appear in the Source Code or Executable Files. 
%
%You agree not to advertise or in any way imply that this Work is a product of Your own. 
%
%The name of the Author may not be used to endorse or promote products derived from the Work without the prior written consent of the Author. 
%
%You agree not to sell, lease, or rent any part of the Work. This does not restrict you from including the Work or any part of the Work inside a larger software distribution that itself is being sold. The Work by itself, though, cannot be sold, leased or rented. 
%
%You may distribute the Executable Files and Source Code only under the terms of this License, and You must include a copy of, or the Uniform Resource Identifier for, this License with every copy of the Executable Files or Source Code You distribute and ensure that anyone receiving such Executable Files and Source Code agrees that the terms of this License apply to such Executable Files and/or Source Code. You may not offer or impose any terms on the Work that alter or restrict the terms of this License or the recipients' exercise of the rights granted hereunder. You may not sublicense the Work. You must keep intact all notices that refer to this License and to the disclaimer of warranties. You may not distribute the Executable Files or Source Code with any technological measures that control access or use of the Work in a manner inconsistent with the terms of this License. 
%
%You agree not to use the Work for illegal, immoral or improper purposes, or on pages containing illegal, immoral or improper material. The Work is subject to applicable export laws. You agree to comply with all such laws and regulations that may apply to the Work after Your receipt of the Work. 
%
%\vparasmall
%Representations, Warranties and Disclaimer. 
%
%THIS WORK IS PROVIDED "AS IS", "WHERE IS" AND "AS AVAILABLE", WITHOUT ANY EXPRESS OR IMPLIED WARRANTIES OR CONDITIONS OR GUARANTEES. YOU, THE USER, ASSUME ALL RISK IN ITS USE, INCLUDING COPYRIGHT INFRINGEMENT, PATENT INFRINGEMENT, SUITABILITY, ETC. AUTHOR EXPRESSLY DISCLAIMS ALL EXPRESS, IMPLIED OR STATUTORY WARRANTIES OR CONDITIONS, INCLUDING WITHOUT LIMITATION, WARRANTIES OR CONDITIONS OF MERCHANTABILITY, MERCHANTABLE QUALITY OR FITNESS FOR A PARTICULAR PURPOSE, OR ANY WARRANTY OF TITLE OR NON-INFRINGEMENT, OR THAT THE WORK (OR ANY PORTION THEREOF) IS CORRECT, USEFUL, BUG-FREE OR FREE OF VIRUSES. YOU MUST PASS THIS DISCLAIMER ON WHENEVER YOU DISTRIBUTE THE WORK OR DERIVATIVE WORKS. 
%
%\vparasmall
%Indemnity. 
%
%You agree to defend, indemnify and hold harmless the Author and the Publisher from and against any claims, suits, losses, damages, liabilities, costs, and expenses (including reasonable legal or attorneys’ fees) resulting from or relating to any use of the Work by You. 
%
%\vparasmall
%Limitation on Liability. 
%
%EXCEPT TO THE EXTENT REQUIRED BY APPLICABLE LAW, IN NO EVENT WILL THE AUTHOR OR THE PUBLISHER BE LIABLE TO YOU ON ANY LEGAL THEORY FOR ANY SPECIAL, INCIDENTAL, CONSEQUENTIAL, PUNITIVE OR EXEMPLARY DAMAGES ARISING OUT OF THIS LICENSE OR THE USE OF THE WORK OR OTHERWISE, EVEN IF THE AUTHOR OR THE PUBLISHER HAS BEEN ADVISED OF THE POSSIBILITY OF SUCH DAMAGES. 
%Termination. 
%
%This License and the rights granted hereunder will terminate automatically upon any breach by You of any term of this License. Individuals or entities who have received Derivative Works from You under this License, however, will not have their licenses terminated provided such individuals or entities remain in full compliance with those licenses. Sections 1, 2, 6, 7, 8, 9, 10 and 11 will survive any termination of this License. 
%If You bring a copyright, trademark, patent or any other infringement claim against any contributor over infringements You claim are made by the Work, your License from such contributor to the Work ends automatically. 
%Subject to the above terms and conditions, this License is perpetual (for the duration of the applicable copyright in the Work). Notwithstanding the above, the Author reserves the right to release the Work under different license terms or to stop distributing the Work at any time; provided, however that any such election will not serve to withdraw this License (or any other license that has been, or is required to be, granted under the terms of this License), and this License will continue in full force and effect unless terminated as stated above. 
%
%\vparasmall
%Publisher. 
%
%The parties hereby confirm that the Publisher shall not, under any circumstances, be responsible for and shall not have any liability in respect of the subject matter of this License. The Publisher makes no warranty whatsoever in connection with the Work and shall not be liable to You or any party on any legal theory for any damages whatsoever, including without limitation any general, special, incidental or consequential damages arising in connection to this license. The Publisher reserves the right to cease making the Work available to You at any time without notice 
%
%\vparasmall
%Miscellaneous 
%
%This License shall be governed by the laws of the location of the head office of the Author or if the Author is an individual, the laws of location of the principal place of residence of the Author. 
%If any provision of this License is invalid or unenforceable under applicable law, it shall not affect the validity or enforceability of the remainder of the terms of this License, and without further action by the parties to this License, such provision shall be reformed to the minimum extent necessary to make such provision valid and enforceable. 
%
%No term or provision of this License shall be deemed waived and no breach consented to unless such waiver or consent shall be in writing and signed by the party to be charged with such waiver or consent. 
%This License constitutes the entire agreement between the parties with respect to the Work licensed herein. There are no understandings, agreements or representations with respect to the Work not specified herein. The Author shall not be bound by any additional provisions that may appear in any communication from You. This License may not be modified without the mutual written agreement of the Author and You. 
%
%
%
%
%
%\newpage
%\subsection{Boost Software License, Version 1.0}
%\label{BoostSoftwareLicense1}
%\begin{center}
%Boost Software License
%Version 1.0 - August 17th, 2003
%\end{center}
%
%Permission is hereby granted, free of charge, to any person or organization
%obtaining a copy of the software and accompanying documentation covered by
%this license (the "Software") to use, reproduce, display, distribute,
%execute, and transmit the Software, and to prepare derivative works of the
%Software, and to permit third-parties to whom the Software is furnished to
%do so, all subject to the following:
%
%\vparasmall
%The copyright notices in the Software and this entire statement, including
%the above license grant, this restriction and the following disclaimer,
%must be included in all copies of the Software, in whole or in part, and
%all derivative works of the Software, unless such copies or derivative
%works are solely in the form of machine-executable object code generated by
%a source language processor.
%
%\vparasmall
%THE SOFTWARE IS PROVIDED "AS IS", WITHOUT WARRANTY OF ANY KIND, EXPRESS OR
%IMPLIED, INCLUDING BUT NOT LIMITED TO THE WARRANTIES OF MERCHANTABILITY,
%FITNESS FOR A PARTICULAR PURPOSE, TITLE AND NON-INFRINGEMENT. IN NO EVENT
%SHALL THE COPYRIGHT HOLDERS OR ANYONE DISTRIBUTING THE SOFTWARE BE LIABLE
%FOR ANY DAMAGES OR OTHER LIABILITY, WHETHER IN CONTRACT, TORT OR OTHERWISE,
%ARISING FROM, OUT OF OR IN CONNECTION WITH THE SOFTWARE OR THE USE OR OTHER
%DEALINGS IN THE SOFTWARE.
%
%
%


\newpage
\subsection{New BSD License (for mpMath)}
\label{New BSD License}
\begin{center}
	New BSD License (for mpMath)
\end{center}

Copyright (c) 2005-2013 Fredrik Johansson and mpmath contributors

All rights reserved.

Redistribution and use in source and binary forms, with or without
modification, are permitted provided that the following conditions are met:

a. Redistributions of source code must retain the above copyright notice,
this list of conditions and the following disclaimer.
b. Redistributions in binary form must reproduce the above copyright
notice, this list of conditions and the following disclaimer in the
documentation and/or other materials provided with the distribution.
c. Neither the name of mpmath nor the names of its contributors
may be used to endorse or promote products derived from this software
without specific prior written permission.


THIS SOFTWARE IS PROVIDED BY THE COPYRIGHT HOLDERS AND CONTRIBUTORS "AS IS"
AND ANY EXPRESS OR IMPLIED WARRANTIES, INCLUDING, BUT NOT LIMITED TO, THE
IMPLIED WARRANTIES OF MERCHANTABILITY AND FITNESS FOR A PARTICULAR PURPOSE
ARE DISCLAIMED. IN NO EVENT SHALL THE REGENTS OR CONTRIBUTORS BE LIABLE FOR
ANY DIRECT, INDIRECT, INCIDENTAL, SPECIAL, EXEMPLARY, OR CONSEQUENTIAL
DAMAGES (INCLUDING, BUT NOT LIMITED TO, PROCUREMENT OF SUBSTITUTE GOODS OR
SERVICES; LOSS OF USE, DATA, OR PROFITS; OR BUSINESS INTERRUPTION) HOWEVER
CAUSED AND ON ANY THEORY OF LIABILITY, WHETHER IN CONTRACT, STRICT
LIABILITY, OR TORT (INCLUDING NEGLIGENCE OR OTHERWISE) ARISING IN ANY WAY
OUT OF THE USE OF THIS SOFTWARE, EVEN IF ADVISED OF THE POSSIBILITY OF SUCH
DAMAGE.





\newpage
\subsection{MIT License}
\label{MITLicense}
\begin{center}
	MIT License
\end{center}

Permission is hereby granted, free of charge, to any person obtaining a copy of this software and associated documentation files (the "Software"), to deal in the Software without restriction, including without limitation the rights to use, copy, modify, merge, publish, distribute, sublicense, and/or sell copies of the Software, and to permit persons to whom the Software is furnished to do so, subject to the following conditions:

\vparasmall
The above copyright notice and this permission notice shall be included in all copies or substantial portions of the Software.

\vparasmall
THE SOFTWARE IS PROVIDED "AS IS", WITHOUT WARRANTY OF ANY KIND, EXPRESS OR IMPLIED, INCLUDING BUT NOT LIMITED TO THE WARRANTIES OF MERCHANTABILITY, FITNESS FOR A PARTICULAR PURPOSE AND NONINFRINGEMENT. IN NO EVENT SHALL THE AUTHORS OR COPYRIGHT HOLDERS BE LIABLE FOR ANY CLAIM, DAMAGES OR OTHER LIABILITY, WHETHER IN AN ACTION OF CONTRACT, TORT OR OTHERWISE, ARISING FROM, OUT OF OR IN CONNECTION WITH THE SOFTWARE OR THE USE OR OTHER DEALINGS IN THE SOFTWARE.





\newpage
\subsection{Excel-DNA License}
\label{ExcelDNALicense}
\begin{center}
	Excel-DNA License
\end{center}

Excel-DNA License
Copyright (C) 2005-2009 Govert van Drimmelen

This software is provided 'as-is', without any express or implied
warranty. In no event will the authors be held liable for any damages
arising from the use of this software.

\vparasmall
Permission is granted to anyone to use this software for any purpose,
including commercial applications, and to alter it and redistribute it
freely, subject to the following restrictions:

\vparasmall
1. The origin of this software must not be misrepresented; you must not
claim that you wrote the original software. If you use this software
in a product, an acknowledgment in the product documentation would be
appreciated but is not required.

2. Altered source versions must be plainly marked as such, and must not be
misrepresented as being the original software.

3. This notice may not be removed or altered from any source distribution.

\vparasmall
Govert van Drimmelen
govert@icon.co.za




%
%
%
%\newpage
%\subsection{AMath License}
%\label{AMathLicense}
%\begin{center}
%AMathLicense
%\end{center}
%
%License
%(C) Copyright 2002-2013 Wolfgang Ehrhardt
%
%\vparasmall
%Copying Conditions
%
%\vparasmall
%This software is provided 'as-is', without any express or implied warranty. In no event will the authors be held liable for any damages arising from the use of this software.
%
%\vparasmall
%Permission is granted to anyone to use this software for any purpose, including commercial applications, and to alter it and redistribute it freely, subject to the following restrictions:
%
%\vparasmall
%1. The origin of this software must not be misrepresented; you must not claim that you wrote the original software. If you use this software in a product, an acknowledgment in the product documentation would be appreciated but is not required.
%
%2. Altered source versions must be plainly marked as such, and must not be misrepresented as being the original software.
%
%3. This notice may not be removed or altered from any source distribution.
%
%
%



\normalsize



%
%
%\chapter{Mathematical Notation} 
%
%%
%%{\bfseries\itshape\sffamily bold italic sans-serif type}
%%
%%{\itshape\sffamily bold italic sans-serif type}
%%
%%{\bfseries\sffamily bold italic sans-serif type}
%%
%%{\sffamily bold italic sans-serif type}
%%
%%
%%$\begin{array}{lcl} z & = & a \\ f(x,y,z) & = & x + y + z \end{array}$
%%
%%$
%%A_{m,n} =
%%\begin{pmatrix}
%%a_{1,1} & a_{1,2} & \cdots & a_{1,n} \\
%%a_{2,1} & a_{2,2} & \cdots & a_{2,n} \\
%%\vdots  & \vdots  & \ddots & \vdots  \\
%%a_{m,1} & a_{m,2} & \cdots & a_{m,n}
%%\end{pmatrix}
%%$
%
%
%\begin{center}
%	\begin{longtable}{p{4cm}  p{12cm}  }
%		\caption{Notation related to Sets}\\
%		%\hline
%		%\noalign{\vskip 1.5mm}
%		%\textbf{Item} & \textbf{Edition 2} \\
%		%\noalign{\vskip 0.8mm}
%		\hline
%		\noalign{\vskip 1mm}
%		\endfirsthead
%		\multicolumn{2}{c}%
%		{\tablename\ \thetable\ -- \textit{Continued from previous page}} \\
%		\hline
%		\noalign{\vskip 1.5mm}
%		\textbf{Item} & \textbf{Edition 2}  \\
%		\noalign{\vskip 0.8mm}
%		\hline
%		\noalign{\vskip 1mm}
%		\endhead
%		\hline \multicolumn{2}{r}{\textit{Continued on next page}} \\
%		\endfoot
%		\hline
%		\endlastfoot
%		$\mathbb{N}$ &	Set of natural numbers	\nomenclature{$\mathbb{N}$}{Set of natural numbers} \\
%		$\mathbb{Z}$ &	Set of integer numbers	\nomenclature{$\mathbb{Z}$}{Set of integer numbers} \\
%		$\mathbb{Q}$ &	Set of rational numbers	\nomenclature{$\mathbb{Q}$}{Set of rational numbers} \\
%		$\mathbb{R}$ &	Set of real numbers	\nomenclature{$\mathbb{R}$}{Set of real numbers} \\
%		$\mathbb{C}$ &	Set of complex numbers	\nomenclature{$\mathbb{C}$}{Set of complex numbers} \\
%	\end{longtable}
%\end{center}
%
%
%\begin{center}
%	\begin{longtable}{p{4cm}  p{12cm}  }
%		\caption{General Notation}\\
%		%\hline
%		%\noalign{\vskip 1.5mm}
%		%\textbf{Item} & \textbf{Edition 2} \\
%		%\noalign{\vskip 0.8mm}
%		\hline
%		\noalign{\vskip 1mm}
%		\endfirsthead
%		\multicolumn{2}{c}%
%		{\tablename\ \thetable\ -- \textit{Continued from previous page}} \\
%		\hline
%		\noalign{\vskip 1.5mm}
%		\textbf{Item} & \textbf{Edition 2}  \\
%		\noalign{\vskip 0.8mm}
%		\hline
%		\noalign{\vskip 1mm}
%		\endhead
%		\hline \multicolumn{2}{r}{\textit{Continued on next page}} \\
%		\endfoot
%		\hline
%		\endlastfoot
%		Required .Net Framework &	Cloudy with rain, across many northern regions. Clear spells across most of Scotland and Northern Ireland,
%		but rain reaching the far northwest.	 \\
%		Windows XP support for .NET & 	yes \\
%		Support for .NET Charts &	no  \\
%		Required .Net Framework &	2	 \\
%		Windows XP support for .NET & 	yes \\
%		Support for .NET Charts &	no  \\
%		Required .Net Framework &	2	 \\
%		Windows XP support for .NET & 	yes \\
%		Support for .NET Charts &	no  \\
%		Required .Net Framework &	2	 \\
%		Windows XP support for .NET & 	yes \\
%		Support for .NET Charts &	no  \\
%		Required .Net Framework &	2	 \\
%		Windows XP support for .NET & 	yes \\
%		Support for .NET Charts &	no  \\
%		Required .Net Framework &	2	 \\
%		Windows XP support for .NET & 	yes \\
%		Support for .NET Charts &	no  \\
%		Required .Net Framework &	2	 \\
%		Windows XP support for .NET & 	yes \\
%		Support for .NET Charts &	no  \\
%		Required .Net Framework &	2	 \\
%		Windows XP support for .NET & 	yes \\
%		Support for .NET Charts &	no  \\
%		Required .Net Framework &	2	 \\
%		Windows XP support for .NET & 	yes \\
%		Support for .NET Charts &	no  \\
%		Required .Net Framework &	2	 \\
%		Windows XP support for .NET & 	yes \\
%		Support for .NET Charts &	no  \\
%	\end{longtable}
%\end{center}
%
